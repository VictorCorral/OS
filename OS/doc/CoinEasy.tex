\documentclass[11pt]{article}
\usepackage{graphics,graphicx}
%\usepackage[dvips]{graphics,graphicx}
\DeclareGraphicsExtensions{.ps,.jpg,.eps,.pdf,.png}
\usepackage{boxedminipage,amsmath,amsfonts}
\usepackage{url}
%\usepackage{secdot}
%\usepackage{natbib}
\usepackage{verbatim}
%\usepackage{moreverb}
\usepackage{enumerate}
\usepackage{makeidx}
\bibliographystyle{plain}
\makeindex

   
%%%%%
% some other macros
\newcommand{\figurepath}{./figures}
\newcommand{\bibpath}{/Users/kmartin/Documents/files/misc}
\newcommand{\figfiletype}{pdf}

%Brad Bell Macros

% Latex macros defined for all the CppAD documentation:
\newcommand{\T}{ {\rm T} }
\newcommand{\R}{ {\bf R} }
\newcommand{\C}{ {\bf C} }
\newcommand{\D}[2]{ \frac{\partial #1}{\partial #2} }
\newcommand{\DD}[3]{ \frac{\partial^2 #1}{\partial #2 \partial #3} }
\newcommand{\Dpow}[2]{ \frac{\partial^{#1}}{\partial  {#2}^{#1}} }
\newcommand{\dpow}[2]{ \frac{ {\rm d}^{#1}}{{\rm d}\, {#2}^{#1}} }

% Define the hangref environment used for the References list:
\newenvironment{hangref}
  {\begin{list}{}{\setlength{\itemsep}{4pt}
  \setlength{\parsep}{0pt}\setlength{\leftmargin}{+\parindent}
  \setlength{\itemindent}{-\parindent}}}{\end{list}}

% Set the page margins to 1 inch all around:
\marginparwidth 0pt\marginparsep 0pt \topskip 0pt\headsep
0pt\headheight 0pt \oddsidemargin 0pt\evensidemargin 0pt
\textwidth 6.5in \topmargin 0pt\textheight 9.0in
\newtheorem{theorem}{Theorem}

   
%%%%Added by Leo%%%%
\newcounter{Fig}
\renewcommand{\theFig}{\arabic{Fig}}
\newcommand{\Fig}[2]{\refstepcounter{Fig} \label{#1}
                     {\small\bf Figure \theFig.} {\small\sl #2 \par}}

\setcounter{topnumber}{3}
\renewcommand{\topfraction}{.9}
\setcounter{bottomnumber}{3}
\renewcommand{\bottomfraction}{.9}
\setcounter{totalnumber}{4}
\renewcommand{\textfraction}{.1}
\setlength{\floatsep}{.25in}
\setlength{\intextsep}{.25in}

\setlength{\fboxrule}{2\fboxrule} \setlength{\fboxsep}{3\fboxsep}

\newcommand{\Sa}{8pt}
\newcommand{\Sb}{0pt}

\renewcommand{\_}{{\char"5F}}
\renewcommand{\{}{{\char"7B}}
\renewcommand{\}}{{\char"7D}}
\renewcommand{\^}{{\char"0D}}

\let\accute= \'
\renewcommand{\'}{{\char"0D}}

\newcommand{\bfit}{\bfseries\itshape}

\newlength{\extopskip} \newlength{\exbottomskip}
\setlength{\exbottomskip}{1\baselineskip}
\addtolength{\exbottomskip}{-5.0pt}
\setlength{\extopskip}{1\exbottomskip}
\addtolength{\extopskip}{-1\parskip}

\newenvironment{Example}{\vspace{1\extopskip}\noindent\hspace*{2em}
                         \frenchspacing\small
                         \tt\begin{tabular}{@{}l@{}}}{
                         \end{tabular}\\[1\exbottomskip]}

\newcommand{\Titem}{\item[$\triangleright$]}
\newcommand{\Ditem}{\item[$\diamond$]}

\newenvironment{Itemize}{\begin{quote}\normalsize
   \baselineskip 20pt plus .3pt minus .1pt \begin{itemize}}
   {\end{itemize}\end{quote}}
   % Set path to folder containing figures
\newcommand{\FigureFolder}{figures}

\newif\ifknitro \knitrofalse    % change to \knitrotrue once we get knitro connected again
\newif\ifipopt  \ipopttrue      % change to \ipopttrue  once we get the build problems sorted out


% We use a number of URLs that point to software downloads. These locations are forever changing,
% and maintaining them is a nightmare. All the URLs are gathered here, so that we can at least
% get away with making changes only once...
\newcommand{\UrlAmpl}{http://www.ampl.com}
\newcommand{\UrlAmplSandia}{http://netlib.sandia.gov/ampl/}
\newcommand{\UrlAmplSolvers}{http://netlib.sandia.gov/cgi-bin/netlib/netlibfiles.tar?filename=netlib/ampl/solvers}
\newcommand{\UrlApacheFileupload}{http://jakarta.apache.org/commons/fileupload/}
\newcommand{\UrlBlas}{ftp://www.netlib.org/blas/blas.tgz}
\newcommand{\UrlBonmin}{http://projects.coin-or.org/Bonmin}
\newcommand{\UrlBuildtools}{http://projects.coin-or.org/BuildTools}
\newcommand{\UrlCbc}{http://projects.coin-or.org/Cbc}
\newcommand{\UrlCgl}{http://projects.coin-or.org/Cgl}
\newcommand{\UrlCl}{http://www.microsoft.com/express/download/\#webInstall}
\newcommand{\UrlClp}{http://projects.coin-or.org/Clp}
\newcommand{\UrlCoinConfig}{https://projects.coin-or.org/BuildTools/wiki/user-configure\#PreparingtheCompilation}
\newcommand{\UrlCoinConfigure}{http://projects.coin-or.org/BuildTools/wiki/user-configure\#CommandLineArgumentsforconfigure}
\newcommand{\UrlCoinCygwin}{http://projects.coin-or.org/BuildTools/wiki/current-issues}
\newcommand{\UrlCoinDownload}{http://projects.coin-or.org/BuildTools/wiki/user-download\#DownloadingtheSourceCode}
\newcommand{\UrlCoinNames}{https://projects.coin-or.org/CoinBinary/wiki/ArchiveNamingConventions}
\newcommand{\UrlCoinProjects}{http://www.coin-or.org/projects/}
\newcommand{\UrlCoinUtils}{http://projects.coin-or.org/CoinUtils}
\newcommand{\UrlCouenne}{http://projects.coin-or.org/Couenne}
\newcommand{\UrlCplex}{http://www.ilog.com/products/cplex/}
\newcommand{\UrlCppad}{http://projects.coin-or.org/CppAD}
\newcommand{\UrlCygwinMake}{http://www.cmake.org/files/cygwin/make.exe}
\newcommand{\UrlCygwinSetup}{http://www.cygwin.com/setup.exe}
\newcommand{\UrlDoxygen}{http://www.doxygen.org}
\newcommand{\UrlDylp}{http://projects.coin-or.org/DyLP}
\newcommand{\UrlEpl}{http://www.eclipse.org/legal/epl-v10.html}
\newcommand{\UrlFToC}{http://www.netlib.org/f2c}
\newcommand{\UrlFToCBin}{http://www.netlib.org/f2c/mswin/}
\newcommand{\UrlFToCZip}{http://www.netlib.org/f2c/libf2c.zip}
\newcommand{\UrlGgs}{http://www.g95.org}
\newcommand{\UrlGamslinks}{https://projects.coin-or.org/svn/GAMSlinks/trunk}
\newcommand{\UrlGcc}{http://gcc.gnu.org}
\newcommand{\UrlGfortran}{http://gcc.gnu.org/fortran/}
\newcommand{\UrlGlpk}{http://www.gnu.org/software/glpk/}
\newcommand{\UrlGlpkDownload}{ftp://ftp.gnu.org/gnu/glpk/glpk-4.42.tar.gz}
\newcommand{\UrlHsl}{http://www.cse.scitech.ac.uk/nag/hsl/}
\newcommand{\UrlIpopt}{http://projects.coin-or.org/Ipopt}
\newcommand{\UrlIpoptDoc}{https://projects.coin-or.org/Ipopt/browser/stable/3.5/Ipopt/doc/documentation.pdf?format=raw}
\newcommand{\UrlIpoptDocxiii}{http://www.coin-or.org/Ipopt/documentation/node13.html}
\newcommand{\UrlKippFileupload}{http://gsbkip.chicagogsb.edu/os/fileupload.html}
\newcommand{\UrlKnitro}{http://www.ziena.com/}
\newcommand{\UrlKnitroMan}{http://www.ziena.com/docs/knitroman.pdf}
\newcommand{\UrlLapack}{ftp://www.netlib.org/lapack/lapack-lite-3.1.0.tgz}
\newcommand{\UrlMingw}{http://downloads.sourceforge.net/mingw/MSYS-1.0.11.exe?modtime=1079444447\&big\_mirror=1}
\newcommand{\UrlMsys}{http://sourceforge.net/project/showfiles.php?group\_id=2435}
\newcommand{\UrlMsysBinary}{http://downloads.sourceforge.net/mingw/bash-3.1-MSYS-1.0.11-1.tar.bz2?modtime=1195140582\&big\_mirror=1}
\newcommand{\UrlMsysAddIns}{http://sourceforge.net/project/showfiles.php?group\_id=2435\&package\_id=67879}
\newcommand{\UrlMsysBison}{bison-2.3-MSYS-1.0.11-1.tar.bz2}
\newcommand{\UrlMsysFlex}{flex-2.5.33-MSYS-1.0.11-1.tar.bz2}
\newcommand{\UrlMsysRegex}{regex-0.12-MSYS-1.0.11-1.tar.bz2}
\newcommand{\UrlNewticket}{http://projects.coin-or.org/OS/newticket}
\newcommand{\UrlNightlyBuild}{https://projects.coin-or.org/TestTools/wiki/NightlyBuildInAction}
\newcommand{\UrlOs}{http://www.optimizationservices.org}
\newcommand{\UrlOsBinaries}{http://www.coin-or.org/download/binary/OS/}
\newcommand{\UrlOsCommon}{https://projects.coin-or.org/svn/OS/branches/OScpp/OSCommon}
\newcommand{\UrlOsDoxygen}{http://www.coin-or.org/OS/doxydoc/html/index.html}
\newcommand{\UrlOsi}{http://projects.coin-or.org/Osi}
\newcommand{\UrlOsJava}{https://projects.coin-or.org/svn/OS/branches/OSjava}
\newcommand{\UrlOsRelease}{https://projects.coin-or.org/svn/OS/releases/2.3.0}
\newcommand{\UrlOsStable}{https://projects.coin-or.org/svn/OS/stable/2.3}
\newcommand{\UrlOsTarball}{http://www.coin-or.org/download/source/OS/}
\newcommand{\UrlOsWiki}{http://projects.coin-or.org/OS/}
\newcommand{\UrlOsWin}{https://projects.coin-or.org/CoinBinary/browser/binary/OS}
\newcommand{\UrlParinclinear}{http://www.coin-or.org/OS/parincLinear.osil}
\newcommand{\UrlSdk}{http://www.microsoft.com/downloads/details.aspx?FamilyID=E6E1C3DF-A74F-4207-8586-711EBE331CDC\&displaylang=en}
\newcommand{\UrlSvn}{http://subversion.tigris.org}
\newcommand{\UrlSymphony}{http://projects.coin-or.org/SYMPHONY}
\newcommand{\UrlTomcat}{http://tomcat.apache.org/}
\newcommand{\UrlTortoiseSvn}{http://tortoisesvn.tigris.org}
\newcommand{\UrlTrac}{http://projects.coin-or.org/OS}
\newcommand{\UrlUsingTrac}{http://www.coin-or.org/usingTrac.html}
\newcommand{\UrlVol}{http://projects.coin-or.org/Vol}
\newcommand{\UrlWget}{http://www.christopherlewis.com/WGet/WGetFiles.htm}
\newcommand{\UrlWgetBinary}{http://www.christopherlewis.com/WGet/wget-1.11.4b.zip}

% Current software versions
\newcommand{\GlpkVer}{4.42}
\newcommand{\OSstable}{2.4}
\newcommand{\OSrelease}{2.4.0}
\newcommand{\OStrunk}{4340}
\newcommand{\MsysVer}{1.0.11}
\newcommand{\MsysFile}{bash-3.1-MSYS-1.0.11}

% Three logicals for allowing the build of three different configurations of the material with slightly different content  
\newif\ifruncode   % to build a manual for folks who only want to run the executables
\newif\ifuselibs   % to build a manual for folks who want to build against the libraries
\newif\ifdevelop   % to build a manual for folks who want to actively develop code

 


\begin{document}

%Some cross-referencing index entries (to be expanded as needed)
\index{AMPL Solver Library |see{Third-party software, ASL}}
\index{ASL|see{Third-party software, ASL}}
\index{Blas|see{Third-party software, Blas}}
\index{Harwell Subroutine Library|see{Third-party software, HSL}}
\index{HSL|see{Third-party software, HSL}}
\index{Lapack|see{Third-party software, Lapack}}
\index{Mumps|see{Third-party software, Mumps}}

\index{Bonmin@{\tt Bonmin}|see{COIN-OR projects, {\tt Bonmin}}}
\index{BuildTools@{\tt BuildTools}|see{COIN-OR projects, {\tt BuildTools}}}
\index{Cbc@{\tt Cbc}|see{COIN-OR projects, {\tt Cbc}}}
\index{Cgl@{\tt Cgl}|see{COIN-OR projects, {\tt Cgl}}}
\index{Clp@{\tt Clp}|see{COIN-OR projects, {\tt Clp}}}
%\index{Configuration Manager|see{Microsoft Visual Studio, Configuration Manager}}
\index{Couenne@{\tt Couenne}|see{COIN-OR projects, {\tt Couenne}}}
\index{CppAD@{\tt CppAD}|see{COIN-OR projects, {\tt CppAD}}}
\index{CoinUtils@{\tt CoinUtils}|see{COIN-OR projects, {\tt CoinUtils}}}
%\index{Debug configuration|see{Microsoft Visual Studio, {\tt Debug} configuration}}
\index{DyLP@{\tt DyLP}|see{COIN-OR projects, {\tt DyLP}}}
\index{GLPK@{\tt GLPK}|see{Third-party software, {\tt GLPK}}}
\index{Ipopt@{\tt Ipopt}|see{COIN-OR projects, {\tt Ipopt}}}
\index{nl files|see{AMPL nl format}}
\index{Osi@{\tt Osi}|see{COIN-OR projects, {\tt Osi}}}
%\index{Release configuration|see{Microsoft Visual Studio, {\tt Release} configuration}}
%\index{Release-plus configuration|see{Microsoft Visual Studio, {\tt Release-plus} configuration}}
\index{SYMPHONY@{\tt SYMPHONY}|see{COIN-OR projects, {\tt SYMPHONY}}}
\index{Vol@{\tt Vol}|see{COIN-OR projects, {\tt Vol}}}

\title{Optimization Services \OSstable\ User's Manual }
\vskip 2in
\author{Horand Gassmann, Jun Ma,  Kipp Martin, and Wayne Sheng}
\maketitle

\begin{abstract}
This is the User's Manual for the Optimization Services (OS) project.  The objective of OS is to provide a
general framework consisting of a set of standards for representing optimization instances, results,
solver options, and communication between clients and solvers in a distributed environment using Web Services.
This COIN-OR\index{COIN-OR} project provides C++ and Java source code for libraries and executable programs that 
implement OS standards.   The OS library includes a robust solver and modeling language interface (API) for linear,
nonlinear and other types of optimization problems.   Also included is the C++ source code for a  command line
executable {\tt OSSolverService}\index{OSSolverService@{\tt OSSolverService}}  for reading problem instances 
(OSiL format\index{OSiL}, nl format\index{AMPL nl format}, MPS format\index{MPS format}) and
calling a solver either locally or on a remote server.  Finally,  both Java\index{Java} source code and a Java {\tt war} 
file are provided for users who wish to set up a solver service on a server running Apache Tomcat\index{Apache Tomcat}.
See the Optimization Services home page {\tt\UrlOs} and the COIN-OR Trac page\index{Trac system} {\tt\UrlTrac} for 
more information.
\end{abstract}


\newpage
\tableofcontents
\listoffigures
\listoftables
\hyphenation{com-plex-Type}
\hyphenation{GAMS-links}



%\noindent\hrulefill
\newpage


\section{The Optimization Services (OS) Project}

The objective of Optimization Services (OS) is to provide a general framework consisting of a set of standards
for representing optimization instances, results, solver options, and communication between clients and solvers
in a distributed environment using Web Services. This COIN-OR project provides source code for libraries and
executable programs that implement OS standards.  See the COIN-OR Trac page {\tt\UrlTrac}\index{Trac system}
or the Optimization Services Home Page {\tt\UrlOs}\index{Optimization Services} for more information.

Like other COIN-OR projects, OS has a versioning system that ensures end users some degree of stability 
and a stable upgrade path as project development continues. The current stable version of OS is \OSstable, 
and the current stable release is \OSrelease\index{OS project!stable release}, based on trunk version~\OStrunk.

\ifruncode
This document provides descriptions for the following components of the OS project:
\else
The OS project provides the following:
\fi

\begin{enumerate}
\item{}  A set of XML\index{XML} based standards for representing optimization instances (OSiL)\index{OSiL}, 
optimization results (OSrL)\index{OSrL}, and optimization solver options (OSoL)\index{OSoL}. 
There are other standards, but these are the main ones. 
The schemas for these standards are described in Section~\ref{section:schemadescriptions}.

\ifruncode\else
\item{}  Open source libraries  that support and implement many of the standards.

\item{}  A robust solver and modeling language interface (API) for linear and nonlinear optimization problems.
Corresponding to the OSiL problem instance representation there is an in-memory object,
{\tt OSInstance}\index{OSInstance@{\tt OSInstance}},
along with a collection of  {\tt get()},   {\tt set()}, and {\tt calculate()} methods for accessing and creating
problem instances. This is a very general API for linear, integer, and nonlinear programs.
Extensions for other major types of optimization problems are also in the works. Any modeling language that can
produce OSiL can easily communicate with any solver that uses the OSInstance API.   
The {\tt OSInstance}\index{OSInstance@{\tt OSInstance}} object
is described in more detail in Section~\ref{section:osinstanceAPI}. The nonlinear part of the API is based on the
COIN-OR project CppAD\index{COIN-OR projects!CppAD@{\tt CppAD}} by Brad Bell ({\tt\UrlCppad}) but is written 
in a very general manner and could be used with other algorithmic differentiation packages. More detail on 
algorithmic differentiation is provided in Section~\ref{section:ad}.
\fi

\item{}  A  command line executable {\tt OSSolverService}\index{OSSolverService@{\tt OSSolverService}}  for reading
problem instances (OSiL format\index{OSiL}, AMPL  nl format\index{AMPL nl format},  
MPS format\index{MPS format}) and calling a solver either locally or on a remote server.
This is described in Section~\ref{section:ossolverservice}.


\item{} Utilities that convert AMPL nl files  and MPS files into the OSiL XML format.
This is described in Section~\ref{section:osmodelinterfaces}.


\item{}  Standards that facilitate the communication between clients and optimization solvers using Web Services.
\ifruncode\else
In  Section~\ref{section:osagent} we describe the {\tt OSAgent}\index{OSAgent@{\tt OSAgent}} part of the OS library
that is used to create Web Services SOAP\index{SOAP protocol} packages with OSiL instances and contact a server for 
solution.
\fi

\item{}  An executable program {\tt OSAmplClient}\index{OSAmplClient@{\tt OSAmplClient}} that is designed to work with 
the AMPL\index{AMPL} modeling language. The {\tt OSAmplClient} appears as a ``solver'' to AMPL and, based on options 
given in AMPL, contacts solvers either remotely or locally to solve instances created in AMPL. This is described in
Section~\ref{section:amplclient}.

\ifruncode\else
\item{}  Server software that works with Apache Tomcat\index{Apache Tomcat} and Apache Axis\index{Apache Axis}.
This software uses Web Services technology and acts as middleware between the client that creates the instance
and the  solver on the server that optimizes the instance and returns the result. This is illustrated in
Section~\ref{section:tomcat}.

\item{}  A lightweight version of the project, {\tt OSCommon},\index{OSCommon@{\tt OSCommon}} for modeling language and 
solver developers who want to use OS API, readers and writers, without the overhead of other COIN-OR projects or any 
third-party software. For information on how to download {\tt OSCommon} see Section~\ref{section:oslite}.
\fi
\end{enumerate}



\section{Quick Roadmap}\label{section:roadmap}

If you want to:

\begin{itemize}
\item Download the OS binaries  (executables and libraries) -- see Section~\ref{section:obtainingbinaries}.

\ifdevelop
\item Download the OS source code -- see Section~\ref{section:downloadsource}.

\item Download just the OS API, readers and writers -- see Section~\ref{section:oslite}.
\fi

\item Use the OSSolverService to read files in nl\index{AMPL nl format}, OSiL\index{OSiL}, 
or MPS format\index{MPS format} and call a solver locally or remotely -- see Section~\ref{section:ossolverservice}.

\item Use modeling languages to generate model instances in OSiL format -- see Section \ref{section:modellang}.

\item Use AMPL\index{AMPL} to solve problems either locally or remotely
with a COIN-OR solver, Cplex\index{cplex@{\tt cplex}},
GLPK\index{Third-party software, {\tt GLPK}}, \ifknitro Knitro\index{knitro}, \fi
or LINDO\index{LINDO} -- see Section~\ref{section:amplclient}.

\item Use GAMS\index{GAMS} to solve problems either locally or remotely -- see Section~\ref{section:gamslinks}.

\ifruncode\else
\item Use MATLAB\index{MATLAB} to generate problem instances in OSiL format 
and call a solver either remotely or locally -- see Section~\ref{section:usingmatlab}.

\item Create your own applications by linking against the binaries -- see Sections \ref{section:examples} and~\ref{section:OSDip}.

\item Use the OS library to build model instances or use solver APIs -- see Sections \ref{section:osmodelinterfaces},
\ref{section:ossolverinterfaces} and~\ref{section:osinstanceAPI}.

\item Use the OS library for algorithmic differentiation\index{Algorithmic differentiation} (in conjunction with 
COIN-OR CppAD)\index{COIN-OR projects!CppAD@{\tt CppAD}} -- see Section~\ref{section:ad}.

\item Build the OS project from the source code -- see Section~\ref{section:build}.
\fi

\ifdevelop
\item Build a remote solver service using Apache Tomcat\index{Apache Tomcat} -- see Section~\ref{section:tomcat}.
\fi
\end{itemize}



\division{Downloading the \ifdevelop OS\else CoinAll\fi  Binaries}\label{section:obtainingbinaries}

\ifdevelop
The OS project is an open-source project  with source code under the Eclipse Public License~(EPL)%
\index{Eclipse Public License (EPL)}.
See~{\tt\UrlEpl}.  This project was initially created by Robert Fourer, Jun Ma, and Kipp Martin.
The code has been written primarily by  Horand Gassmann,   Jun Ma,  and Kipp Martin.    
Horand Gassmann,  Jun Ma,  and Kipp Martin are the COIN-OR project leaders and active developers for the OS project.
\else
The CoinAll project is actually a meta-project consisting of most of the COIN-OR solvers and supporting utility projects.  We describe how to download this project. 
\fi

%Below we describe different methods for obtaining the binaries and C++ source code.
Most users will only be interested in obtaining the binaries, which we describe  next.
%in Section~\ref{section:obtainingbinaries}. The remaining sections of this chapter deal with obtaining %the source code for the project, which will be of interest mostly to developers.
It is also possible to obtain the source code for the project, which will be of interest mostly to developers. 
\ifdevelop
Details can be found in  Section~\ref{section:downloadsource}.
\else
If binaries are not provided for a particular operating system, it may be possible to build them from the source.
For details it is best to start reading the OS web page at~{\tt\UrlOsWiki}.
\fi



%If the user does not wish to compile source code, the OS library, OSSolverService executable
%and Tomcat server software configuration are available in binary format for some operating systems.     
The repository of the binaries is at {\tt\UrlOsBinaries}\index{Downloading!binaries}.
%
\ifdevelop
 Unlike the source code described in Section~\ref{section:downloadwithsvn}, the binary files 
are not subject to version control and can be downloaded using an ordinary browser. 
%If binaries are not provided for a particular operating system,
%it may be possible to build them from the source code. Since the source is under version control, 
%this requires svn. (See Sections \ref{section:svn}, \ref{section:downloadwithsvn} and~\ref{section:build}.)
\fi

The binary distribution for the OS library and executables follows the following naming convention:


\begin{verbatim}
OS-version_number-platform-compiler-build_options.tgz (zip)
\end{verbatim}
For example, OS  Release 2.1.0 compiled with the Intel 9.1 compiler on an Intel 32-bit Linux system is:
\begin{verbatim}
OS-2.1.0-linux-x86-icc9.1.tgz
\end{verbatim}

For more detail on the naming convention and examples see:

\medskip
\noindent{\tt\UrlCoinNames}
\medskip

After unpacking the {\tt tgz} or {\tt zip} archives, the following folders are available.
\begin{itemize}

\item[] {\bf bin --} this directory has the executables {\tt OSSolverService}\index{OSSolverService@{\tt OSSolverService}} 
and {\tt OSAmplClient}\index{OSAmplClient@{\tt OSAmplClient}}.

\item[]  {\bf include --} the header files that are necessary in order to link against the OS library.

\item[] {\bf lib --} the libraries that are necessary for creating applications that use the OS library.

\item[] {\bf  share --} license and author information for all the projects used by the OS project.
\end{itemize}



Files are also provided for an Apache Tomcat\index{Apache Tomcat} Web server along with the associated Web service
that can
read SOAP \index{SOAP protocol} envelopes with model instances in OSiL\index{OSiL} format and/or options in 
OSoL\index{OSoL} format, call the {\tt OSSolverService}\index{OSSolverService@{\tt OSSolverService}},
and return the optimization result in OSrL\index{OSrL} format.
The naming convention\index{file naming conventions} for the server binary is
\begin{verbatim}
OS-server-version_number.tgz (.zip)
\end{verbatim}
For example, the files associated with  OS server release 2.0.0 are in the binary distribution
\begin{verbatim}
OS-server-2.0.0.tgz
\end{verbatim}
There is no platform information given since the server and related binaries were written in Java\index{Java}.
\ifdevelop
The details and use of this distribution are described in Section~\ref{section:tomcat}.
\fi


Finally for Windows users we provide Visual Studio \index{Microsoft Visual Studio} project files 
(and supporting libraries and header files) for building projects based on the OS library and libraries 
used by the OS project. The binary for this is named
\begin{verbatim}
OS-version_number-VisualStudio.zip
\end{verbatim}
For example, the necessary files associated with  OS  stable\index{OS project!stable release} 2.4 
are in the binary distribution
\begin{verbatim}
OS-2.4-VisualStudio.zip
\end{verbatim}
The binaries provided are based on Visual Studio Express 2008.  
\ifdevelop See Section \ref{section:vsexamples} for more detail.\fi


\input{chapters/ossolverservice.tex}

\ifdevelop
\division{OS Support for Modeling Languages, MPS format, Spreadsheets and Numerical Computing Software}\label{section:modellang}
\else
\division{OS Support for AMPL, GAMS and MPS formats}\label{section:modellang}
\fi

Algebraic modeling languages can be used to generate model instances as input to an OS compliant solver.
We describe two such hook-ups, {\tt OSAmplClient} for AMPL\index{AMPL}, and {\tt CoinOS} for
GAMS\index{GAMS} (version 23.8 and above). In addition, we describe the particular version of the 
MPS file format that is supported in {\tt OS}.


\subdivision{AMPL Client:  Hooking AMPL to Solvers}\label{section:amplclient}

\index{OSAmplClient@{\tt OSAmplClient}|(}
\index{AMPL|(}




%This section is based on the assumption that the user has installed  AMPL  on his or her machine.   
It is possible to call all of the COIN-OR solvers 
\ifdevelop
listed in %Section~\ref{section:overview} 
Table~\ref{table:configurations}~(p.\pageref{table:configurations})
\else
that are contained in the CoinAll distribution
\fi
directly from the  AMPL (see {\tt http://www.ampl.com}) modeling language.  In this discussion we assume 
the user has already obtained and installed AMPL.
\ifdevelop  
Both the binary download described in Section~\ref{section:obtainingbinaries}
and the unix and Windows builds (Section \ref{section:unixbuilds}
and~\ref{section:windowsinstall}, respectively) contain
\else
The binary download described in Section~\ref{section:obtainingbinaries}
contains
\fi
%In  the download described in Section~\ref{section:binary} there is 
an executable, {\tt OSAmplClient.exe},
that is linked to all of the COIN-OR solvers 
\ifdevelop
all of the COIN-OR solvers listed in Table~\ref{table:configurations}. %Section~\ref{section:overview}.   
\else
the same solvers as {\tt OSSolverService} described in Section~\ref{section:ossolverservice}.
\fi
From the  perspective of AMPL, the   {\tt OSAmplClient} acts like an AMPL ``solver''.    
The {\tt OSAmplClient.exe}   can be used to solve problems either locally or remotely.   


\subsubdivision{Using OSAmplClient for a Local Solver}\label{section:localampl}

In the following discussion we assume that the AMPL executable {\tt ampl.exe}, the {\tt OSAmplClient},  
and the test problem {\tt  eastborne.mod}\index{eastborne.mod@{\tt eastborne.mod}|(}
 are all in the same directory.  

The  problem instance {\tt eastborne.mod} is an AMPL model file included in the OS distribution 
in the {\tt amplFiles}\index{amplFiles@{\tt amplFiles}} directory.  To solve this problem locally 
by calling {\tt OSAmplClient.exe} from AMPL, first start AMPL and then open the {\tt eastborne.mod} file 
inside AMPL.  The test model {\tt eastborne.mod} is a linear integer program. 


%\begin{verbatim}
%# take in sample integer linear problem
%# assume the problem is in the AMPL directory
\begin{verbatim}
model eastborne.mod;
\end{verbatim}

The next step is to tell AMPL that the solver it is going to use is {\tt OSAmplClient.exe}. 
Do this by issuing the following command inside AMPL.

%\begin{verbatim}
%# tell AMPL that the solver is OSAmplClient
\begin{verbatim}
option solver OSAmplClient;
\end{verbatim}
\ifbible
%
This form of the command assumes that the {\tt OSAmplClient} executable is on the search path. 
If this is not the case, an explicit path to the executable can be given instead, for instance

\begin{verbatim}
option solver "./OSAmplClient";
\end{verbatim}
\fi

It is not necessary to provide the  {\tt OSAmplclient.exe} solver with any options. 
You can just issue the {\tt solve} command in AMPL as illustrated below.  

%\begin{verbatim}
%# solve the problem
\begin{verbatim}
solve;
\end{verbatim}

Of the six methods described in Section~\ref{section:ossolverservice} only the {\tt solve} method 
has been implemented to date.

If no options are specified, the default solver is used, depending on the problem characteristics 
(see Table~\ref{table:defaultsolvers} on p.\pageref{table:defaultsolvers}).\index{default solver}
%is to use {\tt Clp}\index{Clp@{\tt Clp}} for linear programs. 
%For continuous nonlinear models {\tt Ipopt}\index{Ipopt@{\tt Ipopt}} is used. 
%For mixed-integer linear models, {\tt Cbc}\index{Cbc@{\tt Cbc}} is used. 
%For mixed-integer nonlinear models  {\tt Bonmin}\index{Bonmin@{\tt Bonmin}} is used.  
If you wish to specify a specific solver, use the {\tt solver} option.   For example,  
since the test problem {\tt eastborne.mod} is a linear integer program, {\tt Cbc} is used by default. 
If instead you want to  use {\tt SYMPHONY}\index{COIN-OR projects!SYMPHONY@{\tt SYMPHONY}|(},
then you would pass a {\tt solver} option to the {\tt OSAmplclient.exe} solver as follows.%
\index{eastborne.mod@{\tt eastborne.mod}|)}

%\begin{verbatim}
%# tell OSAmplClient to use SYMPHONY instead of Cbc
\begin{verbatim}
option OSAmplClient_options "solver symphony";
\end{verbatim}
\index{COIN-OR projects!SYMPHONY@{\tt SYMPHONY}|)}

Valid values for the {\tt solver} option are installation-dependent.
%{\tt bonmin}, {\tt cbc}, {\tt clp}, {\tt couenne}, {\tt dylp}, {\tt symphony}, and {\tt vol}.   
The solver name in the {\tt solver} option is case insensitive.  

\ifbible
\medskip
It is possible to run {\tt OSAmplClient} in stand-alone mode directly from the command line. In this case, the first command line argument should be the name of a file in .nl format, e.g.,

\begin{verbatim}
OSAmplClient parincQuadratic
\end{verbatim}

\noindent (note that the file extension ({\tt .nl}) is omitted; this information is added by the ASL interface.)
\fi

\subsubdivision{Using OSAmplClient to Invoke an OS Solver Server remotely}\label{section:remoteampl}

Next, assume that you have a large problem you want to solve on a remote solver. It is necessary 
to specify the location of the server solver as an option to OSAmplClient. 
The {\tt serviceLocation} option is used to specify the location of a solver server. 
In this case, the string of options for {\tt OSAmplClient\_options} is:

\begin{verbatim}
serviceLocation  http://xxx/OSServer/services/OSSolverService
\end{verbatim}
where {\tt xxx} is the IP Address for the server. (For instance, Kipp Martin maintains a server that is reachable  at {\tt 74.94.100.129:8080} This string is used to replace the string `{\tt solver symphony}' in the previous example. 
The {\tt serviceLocation} option will send the problem to the %solver server at 
location {\tt http://xxx} and, assuming the remote executable is indeed found 
in the indicated folder, will start the executable.  


\medskip


However, each call 
\begin{verbatim}
option OSAmplClient_options
\end{verbatim}
is memoryless. That is, the options set in the last call will overwrite any options set in previous calls
and cause them to be discarded.  For instance, the sequence of option calls
\begin{verbatim}
option OSAmplClient_options "solver symphony";
option OSAmplClient_options "serviceLocation  
    http://xxx/OSServer/services/OSSolverService";
solve;
\end{verbatim}
will result in the default solver being called. 
If the intent is to use the SYMPHONY solver at the remote location, the option must be declared
as follows:

\begin{verbatim}
option OSAmplClient_options "solver symphony                             \
    serviceLocation http://xxx/OSServer/services/OSSolverService";
solve;
\end{verbatim}


For brevity we will omit the AMPL instruction
\begin{verbatim}
option OSAmplClient_options
\end{verbatim}
the double quotes and the trailing semicolon in the remaining examples.  

\medskip

Finally, the user may wish to pass options to the individual solver. This is done by specifying an options file.
(A sample options file, {\tt solveroptions.osol}\index{solveroptions.osol@{\tt solveroptions.osol}} is 
provided with this distribution).  The name of the options file is the value of the {\tt osol} option.
The string of options to {\tt OSAmplClient\_options} is now
\begin{verbatim}
serviceLocation http://xxx/OSServer/services/OSSolverService              \
    osol solveroptions.osol
\end{verbatim}
This   {\tt solveroptions.osol}  file contains four solver options; two for {\tt Cbc}, one for {\tt Ipopt}, 
and one for {\tt SYMPHONY}\index{COIN-OR projects!SYMPHONY@{\tt SYMPHONY}}.
You can have any number of options. Note the format for specifying an option:
\begin{verbatim}
    <solverOption name="maxN" solver="cbc" value="5" />
\end{verbatim}
The attribute {\tt name} specifies that the option name is {\tt maxN} which is the maximum number of nodes 
allowed in the branch-and-bound tree, the {\tt solver} attribute specifies the name of the solver that the 
option should be applied to, and the {\tt value} attribute specifies the value of the option. 
As a second example, consider the specification
\begin{verbatim}
    <solverOption name="max_iter" solver="ipopt" type="integer" value="2000"/> 
\end{verbatim}
In this example we are specifying an iteration limit for {\tt Ipopt}.  Note the additional attribute 
{\tt type} that has value  {\tt integer}. The Ipopt solver requires specifying the data type 
(string, integer, or numeric) for its options.   Different solvers have different options, 
and we recommend that the user look at the documentation for the solver of interest in order to see 
which options are available.  
A good summary of options for COIN-OR solvers is 
%\url{http://www.coin-or.org/GAMSlinks/gamscoin.pdf}.
\url{http://www.gams.com/dd/docs/solvers/coin.pdf}.

If you examine the file {\tt solveroptions.osol} you will see that there is an XML tag  with the name
{\tt <solverToInvoke>} and that the solver given is {\tt symphony}.   
{\bf This has no effect on a local solve!} However, if this option file is paired with 

\begin{verbatim}
serviceLocation http://xxx/OSServer/services/OSSolverService
osol solveroptions.osol
\end{verbatim}
then in our reference implementation the remote solver service will parse the file {\tt solveroptions.osol}, find the {\tt <solverToInvoke>} tag and then pass the {\tt symphony} solver option to the {\tt OSSolverService} on the remote server.


\ifbible

\subsubdivision{Using OSAmplClient for asynchronous operations}\label{section:amplclient_async}

It is possible to use {\tt OSAmplClient} for asynchronous calls to a remote server, but some additional cautionary remarks are necessary. First, AMPL implements the

\begin{verbatim}
solve;
\end{verbatim}

\noindent command as a sequence of simpler commands, as follows:

\begin{verbatim}
write ('b' & $TMPDIR & '/foo');
shell 'foobar foo -AMPL';
solution foo.sol;
remove foo.nl, foo.sol;
\end{verbatim}

\noindent where {\tt foo} is substituted for a unique name and {\tt foobar} is the name of the solver. The solver options declared in

\begin{verbatim}
option OSAmplClient_options "..."
\end{verbatim}

\noindent are written to an environment variable of the operating system and can be read by the solver program by calling appropriate system routines.

This allows the user to access asynchronous methods as follows.

For an asynchronous {\tt send} command one would specify

\begin{verbatim}
option OSAmplClient_options "-serviceMethod send -serviceLocation ..."
write bfoo; # or gfoo if a text version of the .nl file is desired
shell 'OSAmplClient foo -AMPL';
\end{verbatim}

\noindent omitting the {\tt solution} command, which would cause an AMPL error, since no solution file is returned on a {\tt send} command.

A {\tt retrieve} command could then be invoked by

\begin{verbatim}
option OSAmplClient_options "-serviceMethod retrieve -serviceLocation ..."
shell 'OSAmplClient foo -AMPL';
solution foo.sol;
\end{verbatim}
 

According to David Gay~\cite{dmg-25Sep2012}, ``[f]or the {\tt solution} command to succeed,
such conditions as the numbers of unfixed variables and undropped constraints
and objectives must be the same as when the .nl file behind the .sol file
was written.''
\fi

\subsubdivision{AMPL Summary}

\begin{enumerate}
\item Tell  AMPL to use the OSAmplClient as the solver:

\begin{verbatim}
option solver OSAmplClient;
\end{verbatim}

\item Specify options to the OSAmplClient solver by using the AMPL command 

\begin{verbatim}
option OSAmplClient_options "(option string)";
\end{verbatim}

\item There are three possible options to specify:

\begin{itemize}
\item the location of the options file using  the {\tt osol} option;

\item the location of the remote server using   the {\tt serviceLocation} option;

\item the name of the solver using the  {\tt solver} option; valid values for this option  are 
%{\tt clp}, {\tt cbc},  {\tt dylp},  {\tt ipopt}, {\tt bonmin},   {\tt couenne},  and  {\tt symphony}
installation-dependent. 
For details, see Table~\ref{table:configurations} on page~\pageref{table:configurations} 
and the discussion in Section~\ref{section:OSSolverServiceInputParameters}. 

\end{itemize}

These three options behave {\it exactly like} the {\tt solver}, {\tt serviceLocation}, and {\tt osol} options used by the {\tt OSSolverService} described in  Section \ref{section:commandlineparser}.
Note that the {\tt solver} option only has an effect with a local solve; 
if the user wants to invoke a specific solver with a remote solve, then this must be done in the OSoL file using the {\tt <solverToInvoke>} element.

\item  The options given to {\tt OSAmplClient\_options}  can be given in any order.

\item If no solver is specified using {\tt OSAmplClient\_options},  the default solver is used.
(For details see Table~\ref{table:defaultsolvers}).\index{default solver}

\item A remote solver is called if and only if the {\tt serviceLocation} option is specified.

\end{enumerate}

\index{OSAmplClient@{\tt OSAmplClient}|)}
\index{AMPL|)}



\subdivision{GAMS and Optimization Services}\label{section:gamslinks}

\index{GAMS|(}

This section pertains to GAMS version 23.8 (and above) that now includes support for OS.  
Here we describe the GAMS  implementation of Optimization Services.  We assume that the user has installed GAMS.

In GAMS, OS is implemented through the {\tt CoinOS} solver that is packaged with GAMS.      
The GAMS {\tt CoinOS} solver is really a {\it solver interface} that links to the OS library.
At present the GAMS  {\tt CoinOS} solver does not support local calls, but it can be used to make
remote calls to an {\tt OSSolverService} executable on a remote server. How this is done is the topic of the next section.



\subsubdivision{Using GAMS  to Invoke a Remote OS Solver Service}\label{section:gamsremote}

We now describe how to call  a remote OS   solver service using the GAMS {\tt CoinOS}.  Before proceeding, 
it is important to emphasize that when calling a remote OS solver service, different sets of solvers may be supported, even for the same version of the OS solver service. 
For example, the remote 
implementation may provide access to solvers such as {\tt SYMPHONY}, {\tt Couenne}, {\tt Glpk} and {\tt DyLP}.  
There are several reason why you might wish to use a remote OS solver service. 

\begin{itemize}
\item Have access to a faster machine.

\item  Be able to  submit jobs to run in asynchronous mode -- submit your job,  turn off your laptop,  
and check later to see if the job ran.

\item Call several additional solvers (e.g., {\tt SYMPHONY}, {\tt Couenne}, {\tt Glpk} and {\tt DyLP}).
Note, however, that not all solvers may be available available locally (especially Glpk) may not be available for a remote call.

\end{itemize}

We will illustrate several possible calls with the sample GAMS file {\tt eastborne.gms} which found in the
{\tt  data/gamsFiles} directory. We assume that this file exists in the current directory and that the GAMS executable is found in the search path. The command to execute at the command line would then be

\begin{verbatim} 
gams eastborne.gms MIP=CoinOS optfile=1
\end{verbatim}

The server name ({\tt CoinOS}) is case-insensitive and could equally well have been written as 
``{\tt MIP=coinos}'' or ``{\tt MIP=COINOS}''. Moreover, the file {\tt eastborne.gms} contains the directive

\begin{verbatim}
Option MIP = CoinOS;
\end{verbatim}

\noindent and hence the option {\tt MIP=CoinOS} could have been omitted from the command line.

Since the solver is named {\tt CoinOS}, the options file pointed to by the last part of the command
({\tt optfile=1}) should be named {\tt CoinOS.opt}. In general multiple option files are possible, and the GAMS convention is as follows:

{\tt optfile=1} corresponds to {\tt CoinOS.opt}

{\tt optfile=2} corresponds to {\tt CoinOS.op2}

{\ldots}

{\tt optfile=99} corresponds to {\tt CoinOS.o99}

\medskip
It is important to distinguish between the option files for GAMS just mentioned and the  option file (in OSoL format) passed to the OS solver server (see below).
We now explain the valid options that can go into a GAMS option file when using the CoinOS solver. 
The options are

\vskip 8pt
\noindent{\tt service (string)}: Specifies the URL of  the COIN-OR solver service. 
This option is required in order to direct the remote call appropriately.
\vskip 8pt
Use the following value for this option.
\begin{verbatim}
service http://74.94.100.129:8080/OSServer/services/OSSolverService
\end{verbatim}


\iffalse
\vskip 8pt
\noindent {\tt solver  (string)}:   Specifies the solver that is used to solve an instance. 
Valid values are {\tt clp},  {\tt cbc}, {\tt glpk}, {\tt ipopt},  and {\tt bonmin}.  
If a solver name is specified that is not recognized, the default solver for the problem type is used.  
The value for the solver option is case insensitive. 
For example, if the file {\tt CoinOS.opt} contains the two lines
\begin{verbatim}
service http://74.94.100.129:8080/OSServer/services/OSSolverService
solver glpk
\end{verbatim}
then executing
\begin{verbatim}
gams.exe eastborne.gms optfile 1
\end{verbatim}
will result in  using {\tt Glpk}  to solve the problem.   
\fi

\vskip 8pt
\noindent {\tt writeosil  (string)}:  If this option is used, GAMS will write the optimization instance 
to file {\tt (string)} in    OSiL   format.
\vskip 8pt

\vskip 8pt
\noindent {\tt writeosrl  (string)}:  If this option is used, GAMS will write the result of the optimization 
to file {\tt (string)} in OSrL  format.
\vskip 8pt

The options just described are options for the GAMS modeling language.  
It is also possible to pass options directly to the COIN-OR solvers by using the {\tt OS} interface.
This is done by passing the name of an options file that conforms to the  OSoL  standard.  
%See \url{http://projects.coin-or.org/OS}  for information on Optimization Services.  
The option

\vskip 8pt
\noindent {\tt readosol  (string)}  specifies the name of an OS option  file in OSoL format that is 
given to the solver.  {\bf Note well:} The file  {\tt CoinOS.opt} is an option  file for GAMS but the GAMS option 
{\tt readosol} in the GAMS options file  is specifying the name of an OS options file. 
\vskip 8pt
The file {\tt solveroptions.osol} is contained in the OS distribution in the {\tt osolFiles} directory   
in the {\tt data} directory. This file contains four solver options; two for {\tt Cbc}, one for {\tt Ipopt},
and one for {\tt SYMPHONY} (which is available for remote server calls, but not locally).  
You can have any number of options. Note the format for specifying an option:
\begin{verbatim}
    <solverOption name="maxN" solver="cbc" value="5" />
\end{verbatim}
The attribute {\tt name} specifies that the option name is {\tt maxN} which is the maximum number of nodes 
allowed in the branch-and-bound tree, the {\tt solver} attribute specifies the name of the solver to which
the option should be applied, and the {\tt value} attribute specifies the value of the option. 

Default solver values are present, depending on the problem characteristics. For more details, consult 
Table~\ref{table:defaultsolvers} (p.\pageref{table:defaultsolvers}).
In order to control the solver used, it is necessary to specify the name of the solver
inside the XML tag {\tt <solverToInvoke>}. The example  {\tt solveroptions.osol} file contains the XML tag
\begin{verbatim}
    <solverToInvoke>symphony</solverToInvoke>
\end{verbatim}

\iffalse
If, for example,  the {\tt CoinOS.opt} file is
\begin{verbatim}
solver ipopt
service http://74.94.100.129:8080/OSServer/services/OSSolverService
readosol  solveroptions.osol
writeosrl temp.osrl
\end{verbatim}
then {\tt Ipopt} is ignored as a solver option and the remote server uses the {\tt  SYMPHONY} solver.
\fi  
Valid values for the remote solver service specified in the {\tt <solverToInvoke>} tag are 
installation dependent; the solver service at 
{\tt http://74.94.100.129:8080/OSServer/services/OSSolverService} accepts
{\tt clp},  
{\tt cbc},  {\tt dylp}, {\tt glpk}, {\tt ipopt}, {\tt bonmin},   {\tt couenne},  {\tt symphony}, and 
{\tt vol}.  

%If the problem is solved using a remote solver service the value specified by the 
%GAMS {\tt solver} option is irrelevant and ignored. 

\medskip



By default, the call to the server is a {\it synchronous} call. The GAMS process will wait for the result 
and then display the result. This may not be desirable when solving large optimization models.  
The user may wish to submit a job, turn off his or her computer,  and then check at a later date to see 
if the job is finished.  In order to use the remote solver service in this fashion, i.e., 
{\it asynchronously}, it  is necessary to use the  {\tt service\_method} option.

\vskip 8pt
\noindent {\tt service\_method (string)} specifies the method to execute on a server.  
Valid values for this option are {\tt solve}, {\tt getJobID}, {\tt send}, {\tt knock}, {\tt retrieve}, 
and {\tt kill}. We explain how to use each of these.
\vskip 8pt
The default value of {\tt service\_method} is {\tt solve.} A {\tt solve} invokes the remote service 
in synchronous mode. When using the {\tt solve} method you can optionally specify a set of solver options 
in an OSoL file  by using the {\tt readosol} option. The  remaining values for the {\tt service\_method} 
option are used for an asynchronous call.  We illustrate them in the order in which they would most 
logically be executed. 

\vskip 8pt
\noindent {\tt service\_method getJobID}: When working in asynchronous mode, the server needs to 
uniquely identify each job. The {\tt getJobID} service method will result in the server returning 
a unique job ID. For example if the following {\tt CoinOS.opt} file is used
\vskip 8pt
\begin{verbatim}
service http://74.94.100.129:8080/OSServer/services/OSSolverService
service_method getJobID
\end{verbatim}
with the command
\begin{verbatim}
gams.exe eastborne.gms optfile=1
\end{verbatim}
the user will see a rather long job ID returned to the screen as output. Assume that the job id returned 
is {\tt coinor12345xyz}. This job ID is used to submit a job to the server with the {\tt send} method.
Any job ID can be sent to the server as long as it has not been used before.  

\vskip 8pt
\noindent {\tt service\_method send}: When working in asynchronous mode, use the {\tt send} service method 
to submit a job. When using  the {\tt send} service method a job ID is required. An options file
must be present and must specify a  job ID that has not been used before.  Assume that in the file {\tt CoinOS.opt}  we specify 
the options:
\vskip 8pt
\begin{verbatim}
service http://74.94.100.129:8080/OSServer/services/OSSolverService
service_method send
readosol sendWithJobID.osol
\end{verbatim}
The {\tt sendWithJobID.osol} options file is identical to the {\tt solveroptions.osol} options file except 
that it has an additional XML tag:
\begin{verbatim}
    <jobID>coinor12345xyz</jobID> 
\end{verbatim}
We then execute
\vskip 8pt
\begin{verbatim}
gams.exe eastborne.gms optfile=1
\end{verbatim}
If all goes well, the response to the above command should  be: 
``Problem instance successfully sent to OS service''. 
At this point the server will schedule the job and work on it. It is possible to turn off 
the user computer at this point. At some point the user will want to know if the job is finished. 
This is accomplished using the {\tt knock} service method.
\vskip 8pt
\noindent {\tt service\_method knock}: When working in asynchronous mode, this is used to check the status 
of a job.  Consider the following {\tt CoinOS.opt} file:
\vskip 8pt
\begin{verbatim}
service http://74.94.100.129:8080/OSServer/services/OSSolverService
service_method knock
readosol sendWithJobID.osol 
readospl knock.ospl
writeospl knockResult.ospl
\end{verbatim}
The {\tt knock} service method requires two  inputs. The first input is the name of an options file, 
in this case {\tt sendWithJobID.osol}, specified through the {\tt readosol} option. In addition, a file 
in OSpL format is required. You can use the {\tt knock.opsl} file provided in the binary distribution. 
This file name is specified using the {\tt readospl} option. If no job ID is specified in the OSoL file 
then the status of all jobs on the server will be returned in the file specified by the {\tt writeospl} 
option. If a job ID is specified in the OSoL file, then only information on the specified job ID is 
returned in the file specified by the {\tt writeospl} option.  In this case the file name is 
{\tt knockResult.ospl}. We then execute
\vskip 8pt
\begin{verbatim}
gams.exe eastborne.gms optfile=1
\end{verbatim}
The file {\tt knockResult.ospl} will contain information similar to the following:
\begin{verbatim}
    <job jobID="coinor12345xyz">
        <state>finished</state>
        <serviceURI>http://192.168.0.219:8443/os/OSSolverService.jws</serviceURI>
        <submitTime>2009-11-10T02:13:11.245-06:00</submitTime>
        <startTime>2009-11-10T02:13:11.245-06:00</startTime>
        <endTime>2009-11-10T02:13:12.605-06:00</endTime>
        <duration>1.36</duration>
    </job>
\end{verbatim}
Note that the job is complete as indicated in the {\tt <state>} tag. It is now time to actually retrieve 
the job solution.  This is done with the {\tt retrieve} method.
\vskip 8pt
\noindent {\tt service\_method retrieve}: When working in asynchronous mode, this method is used 
to retrieve the job solution. It is necessary when using {\tt retrieve} %{\tt knock} ???
to specify an options file and in that options file specify a job ID.   
Consider the following {\tt CoinOS.opt} file:
\vskip 8pt
\begin{verbatim}
service http://74.94.100.129:8080/OSServer/services/OSSolverService
service_method retrieve
readosol sendWithJobID.osol
writeosrl answer.osrl
\end{verbatim}
When we then execute
\vskip 8pt
\begin{verbatim}
gams.exe eastborne.gms optfile=1
\end{verbatim}
the result is written to the file {\tt answer.osrl}. 

Finally there is a {\tt kill} service method which is used to kill a job that was submitted by mistake 
or is running too long on the server. 
\vskip 8pt
\noindent {\tt service\_method kill:} When working in asynchronous mode, this method is used to terminate 
a job. You should specify an OSoL  file containing the job ID by using the {\tt readosol} option.
\vskip 8pt

\iffalse
\subsubdivision{Using GAMS to Invoke the Local OS Solver Service \tt CoinOS}\label{section:gamslocal}

   
The GAMS  {\tt CoinOS} solver is really a {\it solver interface} and is linked through the OS library to the 
following COIN-OR solvers: {\tt Bonmin}, {\tt Cbc}, {\tt Clp},  {\tt Glpk}, and {\tt Ipopt}. 
Think of {\tt CoinOS} as a {\it metasolver}.    As an example (we assume a Windows operating system 
and use the .exe extension), consider:

\begin{verbatim}
gams.exe eastborne.gms MIP=CoinOS
\end{verbatim}
The solver name {\tt CoinOS} is not case sensitive and 
\begin{verbatim}
gams.exe eastborne.gms MIP=coinos
\end{verbatim}
will also work.  In addition, if
\begin{verbatim}
Option MIP = CoinOS;
\end{verbatim}
is present in the GAMS file, then writing {\tt MIP=CoinOS} on the command line is unnecessary.
Since {\tt Option MIP = CoinOS;} is present in the GAMS model file {\tt eastborne.gms}, 
we will not specify it explicitly on the command line in the ensuing discussion. To summarize,
\begin{verbatim}
gams.exe eastborne.gms 
\end{verbatim}
is equivalent to the two versions of the command given previously.  Executing any of the commands will 
result in the model being solved on the local machine using the COIN-OR solver {\tt Cbc}, the default solver 
for 
%continuous linear models (LP and RMIP), {\tt CoinOS} chooses {\tt Clp}. For continuous nonlinear 
%models (NLP, DNLP, RMINLP, QCP, RMIQCP), {\tt Ipopt} is the default solver. For 
mixed-integer linear models (MIP).
%,  {\tt Cbc} is the default solver. For mixed-integer nonlinear models (MIQCP, MINLP), 
%{\tt Bonmin} is the default solver.

It is possible to control which solver is selected by {\tt CoinOS}.    This is done by providing an {\it options file}  to  GAMS.   
Since the solver is named {\tt  CoinOS}, the options file should  be named {\tt CoinOS.opt}  (the file name is not case sensitive)
and the command line call is 
\begin{verbatim}
gams.exe eastborne.gms optfile 1
\end{verbatim}
Calling multiple GAMS options files uses the convention

%\begin{verbatim}
{\tt optfile=1} corresponds to {\tt CoinOS.opt}  \\
{\tt optfile=2} corresponds to {\tt CoinOS.op2}  \\
{\tt ...} \\
{\tt optfile=99} corresponds to {\tt CoinOS.o99} \\
%\end{verbatim}

We now explain the valid options that can go into a GAMS option file when using the {\tt CoinOS} solver.  They are:

\vskip 8pt
\noindent {\tt solver  (string)}:   Specifies the solver that is used to solve an instance. 
Valid values are {\tt clp},  {\tt cbc}, {\tt glpk}, {\tt ipopt},  and {\tt bonmin}.  
If a solver name is specified that is not recognized, the default solver for the problem type is used.  
The value for the solver option is case insensitive. 
For example, if the file {\tt CoinOS.opt} contains a single line
\begin{verbatim}
solver glpk
\end{verbatim}
then executing
\begin{verbatim}
gams.exe eastborne.gms optfile 1
\end{verbatim}
will result in  using {\tt Glpk}  to solve the problem.   


\vskip 8pt
\noindent {\tt writeosil  (string)}:  If this option is used, GAMS will write the optimization instance 
to file {\tt (string)} in    OSiL   format.
\vskip 8pt

\vskip 8pt
\noindent {\tt writeosrl  (string)}:  If this option is used, GAMS will write the result of the optimization 
to file {\tt (string)} in OSrL  format.
\vskip 8pt

The options just described are options for the GAMS modeling language.  
It is also possible to pass options directly to the COIN-OR solvers by using the {\tt OS} interface.
This is done by passing the name of an options file that conforms to the  OSoL  standard.  
%See \url{http://projects.coin-or.org/OS}  for information on Optimization Services.  
The option

\vskip 8pt
\noindent {\tt readosol  (string)}  specifies the name of an OS option  file in OSoL format that is 
given to the solver.  Note: The file  {\tt CoinOS.opt} is an option  file for GAMS but the GAMS option 
{\tt readosol} in the GAMS options file  is specifying the name of an OS options file. 
\vskip 8pt
The file {\tt solveroptions.osol} is contained in the OS distribution in the {\tt osolFiles} directory   
in the {\tt data} directory. This file contains four solver options; two for {\tt Cbc}, one for {\tt Ipopt},
and one for {\tt SYMPHONY} (which is available for remote server calls, but not locally).  
You can have any number of options. Note the format for specifying an option:
\begin{verbatim}
    <solverOption name="maxN" solver="cbc" value="5" />
\end{verbatim}
The attribute {\tt name} specifies that the option name is {\tt maxN} which is the maximum number of nodes 
allowed in the branch-and-bound tree, the {\tt solver} attribute specifies the name of the solver to which
the option should be applied, and the {\tt value} attribute specifies the value of the option. 

As a second example, consider the specification
\begin{verbatim}
    <solverOption name="max_iter" solver="ipopt" type="integer" value="2000"/> 
\end{verbatim}
In this example we are specifying an iteration limit for {\tt Ipopt}.  Note the additional attribute 
{\tt type} that has value  {\tt integer}. The Ipopt solver requires specifying the data type 
(string, integer, or numeric) for its options.   For a list of options that solvers take, 
see the file
\begin{verbatim}
docs/solvers/coin.pdf
\end{verbatim}
inside the GAMS directory. 
An up-to-date online version of this list is available at \url{http://www.coin-or.org/GAMSlinks/gamscoin.pdf}.

\fi

\subsubdivision{GAMS Summary:}\label{section:gamssummary}


\begin{enumerate}

\item[1.]   In order to use OS with GAMS you can either specify {\tt CoinOS} as an option to GAMS 
at the command line,
\begin{verbatim}
gams eastborne.gms MIP=CoinOS
\end{verbatim}
or you can  place the statement {\tt Option ProblemType = CoinOS;} somewhere in the model {\it before} 
the {\tt Solve} statement in the GAMS file.


\item[2.]   If no options are given, then the model will be solved locally using the default solver 
(see Table~\ref{table:defaultsolvers} on p.\pageref{table:defaultsolvers}).
%and {\tt Clp} will be used for 
%linear programs, {\tt Cbc} for integer linear programs, {\tt Ipopt} for continuous nonlinear programs, 
%and {\tt Bonmin} for nonlinear integer programs.

\item[3.] In order to control behavior (for example, whether a local or remote solver is used)  an options
 file,  {\tt CoinOS.opt}, must be used as follows

\begin{verbatim}
gams.exe  eastborne.gms optfile=1
\end{verbatim}

\item[4.]  The  {\tt CoinOS.opt} file is used to specify {\it eight potential options}:


\begin{itemize}
\item {\tt service (string)}: using the COIN-OR solver server; this is done by giving the option

\begin{verbatim}
service  http://74.94.100.129:8080/OSServer/services/OSSolverService
\end{verbatim}


\item  {\tt readosol (string)}: whether or not to send the solver an options file; this is done by 
giving the option
\begin{verbatim}
readosol  solveroptions.osol
\end{verbatim}


\item   {\tt solver (string)}: if a local solve is being done,  a specific solver is specified by 
the option
\begin{verbatim}
solver solver_name
\end{verbatim}

Valid values are {\tt clp},  {\tt cbc}, {\tt glpk}, {\tt ipopt} and {\tt bonmin}. %  and {\tt couenne}.  
When the COIN-OR solver service is being used, the only way to specify the solver to use is through 
the {\tt <solverToInvoke>} tag in an OSoL file. In this case the valid values for the solver are  
{\tt clp}, {\tt cbc}, {\tt dylp}, {\tt glpk}, {\tt ipopt}, {\tt bonmin}, {\tt couenne}, {\tt symphony}
and {\tt vol}.



\item  {\tt writeosrl (string)}:  the solution result can be put into an OSrL file by specifying the option

\begin{verbatim}
writeosrl  osrl_file_name
\end{verbatim}



\item    {\tt writeosil (string)}:   the optimization instance  can be put into an OSiL file by specifying 
the option



\begin{verbatim}
writeosil  osil_file_name
\end{verbatim}


\item {\tt writeospl (string):} Specifies the name of an OSpL  file in which the answer from the 
{\tt knock} or {\tt kill} method is written, e.g.,

\begin{verbatim}
writeospl  write_ospl_file_name
\end{verbatim}


\item {\tt readospl (string):} Specifies the name of an OSpL  file that the {\tt knock} method 
sends to  the server

\begin{verbatim}
readospl  read_ospl_file_name
\end{verbatim}

\item {\tt service\_method (string)}: Specifies the method to execute on a server.  Valid values 
for this option are {\tt solve}, {\tt getJobID}, {\tt send}, {\tt knock}, {\tt retrieve}, and {\tt kill}.

\end{itemize}

\item[5.]  If an OS options file is passed to the GAMS {\tt CoinOS} solver using the GAMS  {\tt CoinOS} option      {\tt readosol}, then GAMS does not interpret  or act on any options in this file. The options in the OS options file are passed directly to either: i) the default local solver, ii) the local solver specified by the  GAMS {\tt CoinOS}  option {\tt solver}, or iii)  to the remote OS solver service if one is specified by the GAMS  {\tt CoinOS} option {\tt service.}

\end{enumerate}

\index{GAMS|)}


\subdivision{MPS format}\label{section:MPS}

The MPS format was originally designed for use with IBM's MPS system for solving linear programs%
~\cite{Perrin-MPS}.
In the near half-century since then, extensions, sometimes incompatible, were devised to allow the expression of integer linear programs, special ordered sets, quadratic objectives, etc.

The particular flavor of MPS supported by OS uses the {\tt CoinUtils} library and is described briefly in this section. 

The MPS file is line or record oriented and has three different types of lines:
\begin{enumerate}
\item Header lines, containing a non-blank alphanumeric character in the first position from the left,
\item data lines, which contain a blank character in the first position, and
\item comment lines, which contain an asterisk in the first position.
\end{enumerate}

Header lines separate different sections of the MPS file. Headers that describe portions of the underlying linear programming problem must follow each other in a specific sequence, namely

\begin{verbatim}
NAME
ROWS
COLUMNS
RHS
RANGES
BOUNDS
ENDATA
\end{verbatim}

Only the {\tt NAME}, {\tt ROWS}, {\tt COLUMNS} and {\tt ENDATA} sections are required; the other sections are optional.
Following the {\tt BOUNDS} section and prior to the {\tt ENDATA} record additional optional 
sections can appear, namely

\begin{verbatim}
SOS
QUADOBJ
CSECTION
\end{verbatim}
%BASIS           (Make sure to check with Matt Saltzman before implementing this...)
%\end{verbatim}
  
The order among these three %four 
sections is immaterial. For a description of these sections as well as the content and structure of data records, see~\cite{MPS-MOSEK}. There is also a sample MPS files 
showing all the possibilities in {\tt }

\ifruncode\else    % the matlab interface requires the user to compile stuff

\subdivision{MATLAB:  Using MATLAB to Build and Run OSiL Model Instances}\label{section:usingmatlab}

\index{MATLAB|(}
MATLAB has powerful matrix generation and manipulation routines. This section is for users who wish to use MATLAB to generate the matrix coefficients for linear or quadratic programs and use the OS library to call a solver and get the result back. Using MATLAB with OS requires the user to compile a file {\tt OSMatlabSolverMex.cpp} into a MATLAB executable file (these files will have a {\tt .mex} extension) after compilation. This executable file is linked to the OS library and works through the MATLAB API to communicate with the OS library. 



The OS MATLAB application differs from the other applications in the {\tt OS/applications} folder in that makefiles are not used.  The file 
\begin{verbatim}
OS/applications/matlab/OSMatlabSolverMex.cpp
\end{verbatim}
must be compiled inside the MATLAB command window.  Building the OS MATLAB application requires the following steps. 


\begin{enumerate}[{\bf Step 1:}]



\item{}   The MATLAB installation contains a file {\tt mexopts.sh} (UNIX) or {\tt mexopts.bat}  (Windows) that must be edited.   This file typically resides  in the {\tt bin} directory of the MATLAB application.    This file  contains compile and link options that must be properly set.   Appropriate paths to header files and libraries must be set.  This discussion is based on the assumption that the user has either done a  {\tt make install} for the OS project or has downloaded a binary archive of the OS project. In either case there will be an {\tt include} directory with the necessary header files and a {\tt lib} directory with the necessary libraries for linking. 

First edit   the {\tt CXXFLAGS} option  to point to  the header files in the {\tt cppad} directory and the {\tt include} directory in the project root. For example, it  should look like:
\begin{verbatim}
CXXFLAGS='-fno-common -no-cpp-precomp -fexceptions
    -I/Users/kmartin/Documents/files/code/cpp/OScpp/COIN-OS/
    -I/Users/kmartin/Documents/files/code/cpp/OScpp/COIN-OS/include'
\end{verbatim}

Next edit the {\tt CXXLIBS} flag so that the OS and supporting libraries are included. For example, it should look like the following\footnote{The libraries to include in CXXLIBS depends upon which projects were compiled with OS.} on a MacIntosh:

\begin{verbatim}
CXXLIBS="$MLIBS -lstdc++ -L/Users/kmartin/coin/os-trunk/vpath/lib 
-lOS -lbonmin -lIpopt -lOsiCbc -lOsiClp -lOsiSym -lOsiVol
-lOsiDylp -lCbc -lCgl -lOsi -lClp  -lSym -lVol -lDylp 
-lCoinUtils -lCbcSolver  -lcoinmumps -ldl -lpthread 
/usr/local/lib/libgfortran.dylib -lgcc_s.10.5 -lgcc_ext.10.5 -lSystem -lm 
\end{verbatim}

{\bf Important:} It has been the authors' experience that setting the necessary MATLAB compiler and linker options to build the {\tt mex} can be tricky.  We include in
\begin{verbatim}
OS/applications/matlab/macOSXscript.txt
\end{verbatim}
the exact options that work on a 64 bit Mac with MATLAB release R2009b.

\item{}  Build the MATLAB executable file. Start MATLAB and in the MATLAB command window connect to the directory {\tt OS/applications/matlab} which  contains the file 

\begin{verbatim}
OSMatlabSolverMex.cpp
\end{verbatim}

\item{} Execute the command:

\begin{verbatim}
mex -v OSMatlabSolverMex.cpp
\end{verbatim}

On a 64 bit machine the command should be

\begin{verbatim}
mex -v -largeArrayDims OSMatlabSolverMex.cpp
\end{verbatim}

The name of the resulting executable is system dependent. 
On an Intel MAC OS X 64 bit chip the name will be  {\tt OSMatlabSolver.mexmaci64}, 
on a Windows system it is {\tt OSMatlabSolver.mexw32}.  



\item{}  Set the MATLAB path to include the directory {\tt  OS/applications/matlab}  (or more generally, the directory with the {\tt mex} executable).


\item{}   In the MATLAB command window, connect to the directory {\tt OS/data/matlabFiles}. Run either of the MATLAB
files {\tt markowitz.m} or {\tt parincLinear.m}.  The result should be displayed in the MATLAB browser window.

\end{enumerate}


To use the {\tt OSMatlabSolver} it is necessary to put the coefficients  from a linear, integer, or quadratic problem into MATLAB arrays.   We illustrate for the linear program:

\begin{alignat}{2}
& \mbox{Minimize} & \quad
10 x_{1} + 9 x_{2}\label{eq:parinobj}\\
& \mbox{Subject to} & \quad .7x_{1} + x_{2}  &\le 630  \label{eq:parinccon1}\\
& & .5x_{1} + (5/6) x_{2} &\le 600 \label{eq:parinccon2}\\
& &  x_{1} + (2/3) x_{2} &\le 708 \label{eq:parinccon3}\\
& & .1x_{1} + .25 x_{2} &\le 135 \label{eq:parinccon4}\\
& & x_{1}, x_{2} &\ge 0 \label{eq:parincnonneg}
\end{alignat}

The MATLAB representation of this problem in MATLAB arrays is
\begin{verbatim}
% the number of constraints
numCon = 4;
% the number of variables
numVar = 2;
% variable types
VarType='CC';
% constraint types
A = [.7  1; .5  5/6; 1   2/3; .1   .25];
BU = [630 600  708  135];
BL = [];
OBJ = [10  9];
VL = [-inf -inf];
VU = [];
ObjType = 1;
% leave Q empty if there are no quadratic terms
Q = [];
prob_name = 'ParInc Example'
password = '';
%
%
%the solver
solverName = 'ipopt';
%the remote service address
%if left empty we solve locally -- must solve locally for now
serviceLocation='';
% now solve
callMatlabSolver( numVar, numCon, A, BL, BU, OBJ, VL, VU, ObjType, ...
    VarType, Q, prob_name, password, solverName, serviceLocation)
\end{verbatim}
This example m-file is in the {\tt data} directory and is file {\tt parincLinear.m}. Note that in addition to the problem formulation
we can specify which solver to use through the {\tt solverName} variable.  If solution with a remote solver is desired
this can be specified with the {\tt serviceLocation} variable.  If the {\tt serviceLocation} is left empty, i.e.,
\begin{verbatim}
serviceLocation='';
\end{verbatim}
then a local solver is used. In this case  it is crucial that the appropriate solver is linked in with the {\tt matlabSolver}
executable using the {\tt CXXLIBS} option.


The data directory  also contains the m-file  {\tt template.m} which contains extensive comments about how to formulate
the problems in MATLAB.   The user can edit {\tt template.m} as necessary and create a new instance.




 A second example which is a quadratic problem is given in Section~\ref{section:usingmatlab}.
The appropriate MATLAB m-file is {\tt markowitz.m} in the {data/matlabFiles} directory.
The problem consists in investing  in a number of stocks. The expected returns and risks
(covariances) of the stocks are known. Assume that the decision variables $x_i$
represent the fraction of wealth invested in stock~$i$ and that no stock can have
more than 75\% of the total wealth. The problem then is to minimize the total risk
subject to a budget constraint and a lower bound on the expected portfolio return.

Assume that there are three stocks (variables) and two constraints (not counting the upper limit  %investment
of .75 on the investment variables).


\begin{verbatim}
% the number of constraints
numCon = 2;
% the number of variables
numVar = 3;
\end{verbatim}



All the variables are continuous:


\begin{verbatim}
VarType='CCC';
\end{verbatim}


Next define the constraint upper and lower bounds. There are two constraints, an equality  constraint (an $=$) and a lower bound on portfolio return of .15 (a $\ge$). These two constraints are expressed as



\begin{verbatim}
BL = [1   .15];
BU = [1  inf];
\end{verbatim}



The variables are nonnegative and have upper limits of .75 (no stock can comprise more than 75\% of the portfolio).  This is written as




\begin{verbatim}
VL = [];
VU = [.75 .75 .75];
\end{verbatim}



There are no nonzero linear coefficients in the objective function, but the objective function vector must always be defined and the number of components of this vector is the number of variables.



\begin{verbatim}
OBJ = [0 0 0 ]
\end{verbatim}


 Now the linear constraints.   In the model the two linear constraints are
 \begin{eqnarray*}
 x_{1} + x_{2} + x_{3} &=& 1 \\
 0.3221 x_{1} +   0.0963x_{2} +    0.1187x_{3}  &\ge& .15
 \end{eqnarray*}



 These are expressed as



 \begin{verbatim}
 A = [ 1 1 1  ;
  0.3221   0.0963   0.1187 ];
 \end{verbatim}


Now for the quadratic terms. The only quadratic terms are in the objective function. The objective function is


\begin{eqnarray*}
\min  0.425349694 x_{1}^{2} +  0.445784443 x_{2}^{2} + 0.231430983 x_{3}^{2} + 2 \times 0.185218694 x_{1} x_{2} \\
+ 2 \times 0.139312545 x_{1} x_{3} + 2 \times 0.13881692 x_{2} x_{3}
\end{eqnarray*}


To represent quadratic terms MATLAB uses an array, here denoted $Q$, which has four rows, and a column for each quadratic term. 
In this example there are six quadratic terms. The first row of $Q$ is the row index where the terms appear. By convention, 
the objective function has index -1, and constraints are counted starting at 0.  The first row of $Q$ is


 \begin{verbatim}
 -1 -1 -1 -1 -1 -1
 \end{verbatim}

The second row of $Q$ is the index of the first variable in the quadratic term. We use zero based counting.  
Variable $x_{1}$ has index 0, variable  $x_{2}$ has index 1, and variable $x_{3}$ has index 2.  
Therefore, the second row of $Q$ is



\begin{verbatim}
0 1 2 0 0 1
\end{verbatim}



The third row of $Q$ is the index of the second variable in the quadratic term.   Therefore, the third row of $Q$ is



\begin{verbatim}
0 1 2 1 2 2
\end{verbatim}

Note that terms such as $x_1^2$ are treated as $x_1*x_1$ and that mixed terms such as $x_2x_3$ could be given in either order.

The last (fourth) row is the coefficient. Therefore, the fourth row reads





\begin{verbatim}
.425349654  .445784443  .231430983   .370437388  .27862509   .27763384
\end{verbatim}


The full array is



\begin{verbatim}
Q = [ -1 -1 -1 -1 -1 -1;
      0 1 2 0 0 1 ;
      0 1 2 1 2 2;
      .425349654  .445784443  .231430983   .370437388  .27862509   .27763384
    ];
\end{verbatim}


Finally, name the problem, specify the solver (in this case {\tt ipopt}), the service address (and password if required by the service), and call the solver.



\begin{verbatim}
% replace Template with the name of your  problem
prob_name = 'Markowitz Example from Anderson, Sweeney, Williams, and Martin';
password = '';
%
%the solver
solverName = 'ipopt';
%the remote service service address
%if left empty we solve locally -- must solve locally for now
serviceLocation='';
% now solve
OSCallMatlabSolver( numVar, numCon, A, BL, BU, OBJ, VL, VU, ObjType, VarType, ...
     Q, prob_name, password, solverName, serviceLocation)
\end{verbatim}
\index{MATLAB|)}

\fi




\division{OS Protocols}\label{section:schemadescriptions}

The objective of OS is to provide a set of standards for representing optimization instances, results, solver options,
and communication between clients and solvers in a distributed environment using Web Services.  These standards are
specified by W3C XSD schemas. The schemas for the OS project are contained in the {\tt schemas} folder under the
{\tt OS} root. There are numerous schemas in this directory that are part of the OS standard.
For a full description of all the schemas see  Ma \cite{junma2005}.  We briefly discuss the standards most relevant
to the current version of the OS project.


\subdivision{OSiL (Optimization Services instance Language)} \label{section:osilschema}
OSiL\index{OSiL|(} is
an XML-based language for representing instances of large-scale
optimization problems including linear programs, mixed-integer programs,
quadratic programs, and very general nonlinear programs.

OSiL stores optimization problem instances as XML files.  Consider the following problem instance, which is a
modification of an example of Rosenbrock\index{Rosenbrock, H.H.@{\it Rosenbrock, H.H.}}~\cite{rosenbrock1960}:
%
\begin{alignat}{2}
& \mbox{Minimize} & \quad (1 - x_{0})^{2} + 100(x_{1} - x_{0}^{2})^{2} + 9x_{1} \label{eq:roobj}\\
& \mbox{s.t.} & \quad x_{0} + 10.5 x_{0}^{2} + 11.7 x_{1}^{2} + 3x_{0}x_{1}  &\le 25  \label{eq:ro1}\\
& & \ln(x_{0} x_{1}) + 7.5 x_{0} + 5.25 x_{1} &\ge 10 \label{eq:ro2}\\
& & x_{0}, x_{1} &\ge 0 \label{eq:ro3}
\end{alignat}


There are two continuous variables, $x_{0}$ and $x_{1}$, in this instance, each with a lower bound of 0.
Figure~\ref{figure:variableselement} shows how we represent this information in an XML-based OSiL file.
Like all XML files, this is a text file that contains both {\it markup} and {\it data}. In this case there
are two types of markup, {\it elements} (or {\it tags}\/) and {\it attributes} that describe the elements.
Specifically, there are a {\tt <variables>} element and two {\tt <var>} elements. Each {\tt <var>}
element has attributes {\tt lb}, {\tt name}, and {\tt type} that
describe properties of a decision variable: its lower bound, ``name'', and
domain type (continuous, binary, general integer).


\begin{figure}[b]
\centering
   \small {\obeyspaces\let =\
\fbox{\tt\begin{tabular}{@{}l@{}}
<variables numberOfVariables="2">\\[\Sb]
    <var lb="0" name="x0" type="C"/>\\[\Sb]
    <var lb="0" name="x1" type="C"/>\\[\Sb]
</variables>\\[\Sb]
\end{tabular} }} \medskip
\caption{The {\tt <variables>} element for the example (1)--(4).}\label{figure:variableselement}
\end{figure}


     To be useful for communication between solvers and modeling
languages, OSiL instance files must conform to a standard.
An XML-based representation standard is imposed
through the use of a {\em W3C XML Schema.} The W3C, or World Wide
Web Consortium (\url{www.w3.org}), promotes standards for
the evolution of the web and for interoperability between web
products.  XML Schema (\url{www.w3.org/XML/Schema}) is one
such standard.  A schema specifies the elements and attributes that
define a specific XML vocabulary. The W3C XML Schema is thus a schema
for schemas; it specifies the elements and attributes for a schema
that in turn specifies elements and attributes for an XML
vocabulary such as OSiL. An XML file that conforms to a
schema is called {\it valid} for that schema.

     By analogy to object-oriented programming, a schema is akin to a header file in C++ that defines the members and methods in a class.  Just as a class in C++ very explicitly describes member and method names and properties, a
schema explicitly describes element and attribute names and properties.

{\small
\begin{figure}[b]
   \small {\obeyspaces\let =\
\makebox[0in][t]{\fbox{\tt\begin{tabular}{@{}l@{}}
<xs:complexType name="Variables">\\[\Sb]
    <xs:sequence>\\[\Sb]
        <xs:element name="var" type="Variable" maxOccurs="unbounded"/>\\[\Sb]
    </xs:sequence>\\[\Sb]
    <xs:attribute name="numberOfVariables"\\[\Sb]
            type="xs:positiveInteger" use="required"/>\\[\Sb]
</xs:complexType>\\[\Sb]
\end{tabular} }}} \medskip
\caption{The {\tt  Variables} complexType  in the OSiL
schema.}\label{figure:osilvariables}
\end{figure}
}%end small


{\small
\begin{figure}[b]
   \small {\obeyspaces\let =\
\makebox[0in][t]{\fbox{\tt\begin{tabular}{@{}l@{}}
<xs:complexType name="Variable">\\[\Sb]
    <xs:attribute name="name" type="xs:string" use="optional"/>\\[\Sb]
    <xs:attribute name="init" type="xs:string" use="optional"/>\\[\Sb]
    <xs:attribute name="type" use="optional" default="C">\\[\Sb]
        <xs:simpleType>\\[\Sb]
            <xs:restriction base="xs:string">\\[\Sb]
                <xs:enumeration value="C"/>\\[\Sb]
                <xs:enumeration value="B"/>\\[\Sb]
                <xs:enumeration value="I"/>\\[\Sb]
                <xs:enumeration value="S"/>\\[\Sb]
            </xs:restriction>\\[\Sb]
        </xs:simpleType>\\[\Sb]
    </xs:attribute>\\[\Sb]
    <xs:attribute name="lb" type="xs:double" use="optional" default="0"/>\\[\Sb]
    <xs:attribute name="ub" type="xs:double" use="optional" default="INF"/>\\[\Sb]
</xs:complexType>\\[\Sb]
\end{tabular} }}} \medskip
\caption{The {\tt  Variable} complexType in the OSiL
schema.}\label{figure:osilvar}
\end{figure}
} %end small



Figure~\ref{figure:osilvariables} is a piece of our schema for OSiL. In W3C XML Schema jargon, it defines a {\it complexType,}  whose purpose is to specify elements and attributes that are allowed to appear in a valid XML instance file such as the one excerpted in Figure~\ref{figure:variableselement}. In particular, Figure~\ref{figure:osilvariables} defines the complexType named {\tt Variables}, which
comprises an element named {\tt <var>} and an attribute named {\tt
numberOfVariables}. The {\tt numberOfVariables} attribute is of a
standard type {\tt positiveInteger}, whereas the {\tt <var>} element is
a user-defined complexType named {\tt Variable}. Thus the complexType {\tt
Variables} contains a sequence of {\tt <var>} elements that
are of complexType {\tt Variable}. OSiL's schema must also provide a
specification for the {\tt Variable} complexType, which is shown in
Figure~\ref{figure:osilvar}.

In OSiL the linear part of the problem is stored in the  {\tt
<linearConstraintCoefficients>} element, which stores the coefficient
matrix using three arrays as proposed in the earlier LPFML schema
\cite{fourer2005a}.  There is a child element of {\tt <linearConstraintCoefficients>} 
to represent each array: {\tt <value>} for an array of nonzero coefficients, 
{\tt <rowIdx>} or {\tt <colIdx>} for a corresponding array of row indices or column indices, 
and {\tt <start>} for an array that indicates where each row or column begins in the previous two arrays.
This is shown in Figure~\ref{figure:rowlistMatrix}.


\begin{figure}[ht]
\centering
   \small {\obeyspaces\let =\
\fbox{\tt\begin{tabular}{@{}l@{}}
<linearConstraintCoefficients numberOfValues="3">\\[\Sb]
    <start>\\[\Sb]
        <el>0</el><el>2</el><el>3</el>\\[\Sb]
    </start>\\[\Sb]
    <rowIdx>\\[\Sb]
        <el>0</el><el>1</el><el>1</el>\\[\Sb]
    </rowIdx>\\[\Sb]
    <value>\\[\Sb]
        <el>1.</el><el>7.5</el><el>5.25</el>\\[\Sb]
    </value>\\[\Sb]
</linearConstraintCoefficients>\\[\Sb]
\end{tabular} }} \medskip\\[\Sb]
\caption{The {\tt <linearConstraintCoefficients>} element for constraints
(\ref{eq:ro1}) and (\ref{eq:ro2}).}\label{figure:rowlistMatrix}
\end{figure}

The quadratic part of the problem is represented  in Figure~\ref{figure:qterms}.

\begin{figure}[ht]
\centering
   \small {\obeyspaces\let =\
\fbox{\tt\begin{tabular}{@{}l@{}}
<quadraticCoefficients numberOfQuadraticTerms="3">\\[\Sb]
     <qTerm idx="0" idxOne="0" idxTwo="0" coef="10.5"/>\\[\Sb]
     <qTerm idx="0" idxOne="1" idxTwo="1" coef="11.7"/>\\[\Sb]
     <qTerm idx="0" idxOne="0" idxTwo="1" coef="3."/>\\[\Sb]
</quadraticCoefficients>\\[\Sb]
\end{tabular} }} \medskip
\caption{The {\tt <quadraticCoefficients>} element for constraint (\ref{eq:ro1}).}
\label{figure:qterms}
\end{figure}

The nonlinear part of the problem is given in Figure~\ref{figure:roobjnlnode}.



{\small
\begin{figure}[t]
\centering
   \small {\obeyspaces\let =\
\fbox{\tt\begin{tabular}{@{}l@{}}
<nl idx="-1">\\[\Sb]
     <plus>\\[\Sb]
          <power>\\[\Sb]
               <minus>\\[\Sb]
                    <number value="1.0"/>\\[\Sb]
                    <variable coef="1.0" idx="0"/>\\[\Sb]
               </minus>\\[\Sb]
               <number value="2.0"/>\\[\Sb]
          </power>\\[\Sb]
          <times>\\[\Sb]
               <power>\\[\Sb]
                    <minus>\\[\Sb]
                         <variable coef="1.0" idx="0"/>\\[\Sb]
                         <power>\\[\Sb]
                              <variable coef="1.0" idx="1"/>\\[\Sb]
                              <number value="2.0"/>\\[\Sb]
                         </power>\\[\Sb]
                    </minus>\\[\Sb]
                    <number value="2.0"/>\\[\Sb]
               </power>\\[\Sb]
               <number value="100"/>\\[\Sb]
          </times>\\[\Sb]
     </plus>\\[\Sb]
</nl>\\[\Sb]
\end{tabular} }} \medskip\\[\Sb]
\caption{The {\tt <nl>} element for the nonlinear part of the objective (\ref{eq:roobj}).}\label{figure:roobjnlnode}
\end{figure}
}

The complete OSiL representation can be found in the Appendix (Section~\ref{section:rosenbrockXML}).%
\index{OSiL|)}

\subdivision{OSnL (Optimization Services nonlinear Language)} \label{section:osnlschema}
The OSnL\index{OSnL|(} schema is imported by the OSiL\index{OSiL} schema and is used 
to represent the nonlinear part of an optimization instance. 
This is explained in greater detail in \ifruncode the OS User's Manual\else Section~\ref{section:osexpressiontreeclass}\fi. Also refer to
Figure~\ref{figure:roobjnlnode} for an illustration of elements from the OSnL standard. This figure represents
the nonlinear part of the objective in equation~(\ref{eq:roobj}), that is,
%
$$
(1-x_0)^2 + 100 (x_1-x_0^2)^2.
$$
\index{OSnL|)}


\subdivision{OSrL (Optimization Services result Language)} \label{section:osrlschema}
OSrL\index{OSrL|(} is an XML-based language for representing the solution of large-scale
optimization problems including linear programs, mixed-integer programs,
quadratic programs, and very general nonlinear programs.  An example solution (for the problem given in
 (\ref{eq:roobj})--(\ref{eq:ro3}) ) in OSrL format is given below.

{\small
\begin{verbatim}
<?xml version="1.0" encoding="UTF-8"?>
<?xml-stylesheet type = "text/xsl"
  href = "/Users/kmartin/Documents/files/code/cpp/OScpp/COIN-OSX/OS/stylesheets/OSrL.xslt"?>
<osrl xmlns="os.optimizationservices.org"
      xmlns:xsi="http://www.w3.org/2001/XMLSchema-instance"
      xsi:schemaLocation="os.optimizationservices.org
      http://www.optimizationservices.org/schemas/2.0/OSiL.xsd">
    <general>
        <generalStatus type="normal"/>
        <serviceName>Solved using a LINDO service</serviceName>
        <instanceName>Modified Rosenbrock</instanceName>
    </general>
    <optimization numberOfSolutions="1" numberOfVariables="2" numberOfConstraints="2"
        numberOfObjectives="1">
        <solution targetObjectiveIdx="-1">
            <status type="optimal"/>
            <variables>
                <values numberOfVar="2">
                    <var idx="0">0.87243</var>
                    <var idx="1">0.741417</var>
                </values>
                <other numberOfVar="2" name="reduced_costs" description="the variable reduced costs">
                    <var idx="0">-4.06909e-08</var>
                    <var idx="1">0</var>
                </other>
            </variables>
            <objectives>
                <values numberOfObj="1">
                    <obj idx="-1">6.7279</obj>
                </values>
            </objectives>
            <constraints>
                <dualValues numberOfCon="2">
                    <con idx="0">0</con>
                    <con idx="1">0.766294</con>
                </dualValues>
            </constraints>
        </solution>
    </optimization>
\end{verbatim}
}
% Hide this stuff for now...
% The OSrL schema is also used to return timer and system statistics that are sometimes 
% gathered by the solvers themselves or generated as a result of using the {\tt knock} 
% method. (See the example given in Section~\ref{section:knock}.)
\index{OSrL|)}



\subdivision{OSoL (Optimization Services option Language)} \label{section:osolschema}
OSoL\index{OSoL|(} is
an XML-based language for representing options that get passed to an optimization solver or a hosted optimization
solver Web service. It contains both standard options for generic services and extendable option tags for
solver-specific directives.
Several examples of files in OSoL format are presented in Section~\ref{section:servicemethods}.%
\index{OSoL|)}

\subdivision{OSpL (Optimization Services process Language)} \label{section:osplschema}
\index{OSpL|(}This is a standard used to enquire about dynamic process information that 
is kept by the Optimization Services registry. The string passed to the {\tt knock} 
method is in the OSpL format. See the example given in Section~\ref{section:knock}.\index{OSpL|)}



\division{The  OSInstance API}\label{section:osinstanceAPI}

The OSInstance API can be used to:

\begin{itemize}

\item  get information about model parameters, or convert the {\tt OSExpressionTree} into a prefix or postfix
representation through a collection  of {\tt get()} methods,

\item modify, or even create an instance from scratch, using a number of {\tt set()} methods,

\item provide information to solvers that require function evaluations, Jacobian and Hessian sparsity patters,  
function gradient evaluations, and Hessian evaluations.

\end{itemize}



\subdivision{Get Methods}

The {\tt get()} methods are used by other classes to access data in an existing {\tt OSInstance} object or get 
an expression tree representation of an instance in postfix or prefix format.   Assume {\tt osinstance} is an 
object in the {\tt OSInstance} class created as illustrated in Figure~\ref{figure:creatingosinstanceobject}. 
Then, for example,
\begin{verbatim}
osinstance->getVariableNumber();
\end{verbatim}
will return an integer which is the number of variables in the problem,
\begin{verbatim}
osinstance->getVariableTypes();
\end{verbatim}
will return a {\tt char} pointer to the variable types ({\tt C} for continuous, {\tt B} for binary, 
and {\tt I} for general integer),
\begin{verbatim}
getVariableLowerBounds();
\end{verbatim}
will  return a {\tt double} pointer to the lower bound on each variable. There are similar {\tt get()} methods 
for the constraints. There are numerous {\tt get()} methods for the data in the {\tt <linearConstraintCoefficients>} 
 element, the {\tt <quadraticCoefficients>} element, and the {\tt <nonlinearExpressions>} element.

When an {\tt osinstance} object is created, it is stored as an expression tree in an {\tt OSExpressionTree} object. 
However, some solver APIs (e.g., LINDO) may take the data in a different format such as postfix and prefix. 
There are methods to return the data in either postfix or prefix format.

First define a {\tt vector} of pointers to {\tt OSnLNode} objects.
\begin{verbatim}
std::vector<OSnLNode*> postfixVec;
\end{verbatim}
then get the expression tree for the objective function (index = -1) as a postfix vector of nodes.
\begin{verbatim}
postfixVec = osinstance->getNonlinearExpressionTreeInPostfix( -1);
\end{verbatim}
If, for example, the {\tt osinstance} object was the in-memory representation of   the instance illustrated 
in  Section~\ref{section:rosenbrockXML} and Figure~\ref{figure:expressiontree} then the code
\begin{verbatim}
for (i = 0 ; i < n; i++){
    cout << postfixVec[i]->snodeName << endl;
}
\end{verbatim}
will produce
\begin{verbatim}
number
variable
minus
number
power
number
variable
variable
number
power
minus
number
power
times
plus
\end{verbatim}

This postfix traversal of the expression tree in Figure~\ref{figure:expressiontree} lists all the nodes
by recursively processing all subtrees, followed by the root node.
The method {\tt processNonlinearExpressions()} in the {\tt LindoSolver} class in the {\tt OSSolverInterfaces} 
library component illustrates the use of a postfix vector of {\tt OSnLNode} objects to build a Lindo model instance.


\subdivision{Set Methods}

The {\tt set()} methods can be used to build an in-memory {\tt OSInstance}
 object. A code example of how to do this is in Section~\ref{section:exampleOSInstanceGeneration}.

\subdivision{Calculate Methods}

The {\tt calculate()} methods are described in Section~\ref{section:ad}.


\subdivision{Modifying an   {\tt OSInstance} Object}\label{section:osinstanceMod}

The OSInstance API is designed to be used to either build an in-memory {\tt OSInstance} object 
or provide information about the in-memory object (e.g., the number of variables).   
This interface is not designed for problem modification.  We plan on later providing an {\tt OSModification} 
object for this task. However, by directly accessing an {\tt OSInstance} object it is possible 
to modify parameters in the following classes:

\begin{itemize}
\item {\tt Variables}

\item {\tt Objectives}

\item {\tt Constraints}

\item {\tt LinearConstraintCoefficients}
\end{itemize}

For example, to modify the first nonzero objective function coefficient of the first objective  function to 10.7 the user would write,

\begin{verbatim}
osinstance->instanceData->objectives->obj[0]->coef[0]->value = 10.7;
\end{verbatim}
If the user wanted to modify the actual number of nonzero coefficients as declared by 
\begin{verbatim}
osinstance->instanceData->objectives->obj[0]->numberOfObjCoef;
\end{verbatim}
then the only safe course of action would be to delete the current {\tt OSInstance} object 
and build a new one  with the modified coefficients. It is strongly recommend that no changes 
are made involving allocated memory -- i.e., any kind of {\tt numberOf***}.  
Modifying an objective function coefficient is illustrated in the OSModDemo example. 
See Section \ref{section:exampleOSModDemo}.

After modifying an {\tt OSInstance} object, it is necessary to set certain boolean variables 
to true in order for these changes to get reflected in the OS solver interfaces.

\begin{itemize}
\item {\tt Variables} -- if any changes are made to a parameter in this class set

\begin{verbatim}
osinstance->bVariablesModified = true;
\end{verbatim}

\item {\tt Objectives} -- if any changes are made to a parameter in this class set

\begin{verbatim}
osinstance->bObjectivesModified = true;
\end{verbatim}

\item {\tt Constraints} -- if any changes are made to a parameter in this class set

\begin{verbatim}
osinstance->bConstraintsModified = true;
\end{verbatim}

\item {\tt LinearConstraintCoefficients} -- if any changes are made to a parameter in this class set

\begin{verbatim}
osinstance->bAMatrixModified = true;
\end{verbatim}
\end{itemize}

At this point, if the user desires to modify an {\tt OSInstance} object that contains nonlinear terms, 
the only safe strategy is to delete the object and build a new object that contains the modifications. 



\subdivision{Printing a Model for Debugging}\label{section:printModel}

The OSiL representation for the test problem {\tt rosenbrockmod.osil} is given in 
Appendix~\ref{section:rosenbrockXML}.  Many users will not find the OSiL representation 
useful for model debugging purposes.  For users who wish to see a model in a standard infix 
representation we provide a method {\tt printModel()}.  Assume that we have an {\tt osinstance} 
object in the {\tt OSInstance} class that represents the model of interest.  The call
\begin{verbatim}
osinstance->printModel( -1)
\end{verbatim}
will result in printing the (first) objective function indexed by -1.  In order to print 
constraint~$k$ use
\begin{verbatim}
osinstance->printModel( k)
\end{verbatim}
In order to print the entire model use
\begin{verbatim}
osinstance->printModel( )
\end{verbatim}

 
Below we give the result of {\tt osintance->printModel( )} for the problem {\tt rosenbrockmod.osil}.
\begin{verbatim}
Objectives:
min 9*x_1 + (((1 - x_0) ^ 2) + (100*((x_1 - (x_0 ^ 2)) ^ 2)))

Constraints:
(((((10.5*x_0)*x_0) + ((11.7*x_1)*x_1)) + ((3*x_0)*x_1)) + x_0) <= 25  
10 <= ((ln( (x_0*x_1)) + (7.5*x_0)) + (5.25*x_1))

Variables:
x_0 Type = C  Lower Bound =  0  Upper Bound =  1.7976931348623157e308
x_1 Type = C  Lower Bound =  0  Upper Bound =  1.7976931348623157e308
\end{verbatim}
 



\addcontentsline{toc}{section}{Bibliography}
% \addcontentsline{toc}{section}{Bibliography}% alternative for article class
%\bibliography{\bibpath/kippbib}
\bibliographystyle{amsplain}
\bibliography{kippbib}
 
\printindex
  
\end{document}

%===========================================================================================================
%
%  This is the end of the document. Below are some private things we like to keep for future reference...
%
%===========================================================================================================

Here is where you get winsock.h
 
 
ftp://sunsite.unc.edu/pub/micro/pc-stuff/ms-windows/winsock/winsock-1.1/winsock.h

see also

ftp://sunsite.unc.edu/pub/micro/pc-stuff/ms-windows/winsock/winsock-1.1/


For the SDK (windows.h)

http://www.microsoft.com/downloads/details.aspx?FamilyId=A55B6B43-E24F-4EA3-A93E-40C0EC4F68E5&displaylang=en#Instructions

   
Important comments from Lou:

	Using Coin-All as of late afternoon (up-to-date, according to svn
update), and after installing the SDK as per Andreas' email, I have a `minimal
Msys' build using cl, with ASL, and OS, but it's definitely not clean. Here are
the issues, and the workarounds I used.

	Cbc.exe doesn't link with ampl in the mix, and I put the blame on
amplsolv.lib. MS link reports a conflict, apparently libcmt.lib and libcmtd.lib
(normal and debugging malloc, respectively) are both pulled in. There's a fix:
The link message says `use NODEFAULTLIB', and sure enough, if I edit the
Makefile to read

 $(CXXLINK) $(cbc_LDFLAGS) $(cbc_OBJECTS) $(cbc_LDADD) $(LIBS) \
   "-link -NODEFAULTLIB:libcmt.lib"

then cbc.exe will link. The quotes are necessary because the cl mechanism for
passing things to the linker is `everything on the line after -link', and
without the quotes libtool rearranges things. I don't know enough about MSVS
defaults to say whether libcmtd is the default on a debug build, or libcmt. JP
might have some insight. I have no immediate ideas on how to work this hack
through autotools from a Makefile.am. Might be easier to fix amplsolv.lib.

	The same hack is necessary to link OSSolverService.exe, the OS
unitTest.exe, and the OS OSAmplClient.exe.

	When building OSAgent, the -I/include and -I/mingw/include have to go.
The cl include order is `current directory, -I options, INCLUDE environment
variable.' When either of /include or /mingw/include are specified with -I,
MinGW include files are selected first, and they're not compatible with MS
include files.

	I didn't run into the problem Ted describes, but that's probably
because I'm doing static links and the function (CbcOrClpReadCommand) is never
used anywhere except in cbc.exe. Link is bright enough to know it doesn't need
it.

	The Alps unit test crashes. Which is odd because the Bcps and Blis unit
tests run just fine.

	The OS unit test runs until it gets to the Ampl testing section,
and dies because it can't find parinc.nl. I'm not an Ampl user --- is a .nl
file Ampl output? If so, the result makes sense, because I don't have Ampl on
this machine.

	While we're here, the -rpath <path> going into libtool seems to
translate into an attempt to pass -L<path> to cl, which it ignores with a
warning about an invalid option.  Whatever libtool is hoping to achieve, it's
failing. And not necessary, apparently, at least for a static build.

%%%%

Important comments from Andreas:



This is a well-known problem with the free cl compiler.  I had the same problem and search the web.

This is what I believe I did:

Download and install the (free) SDK, which has the missing header files:

http://www.microsoft.com/downloads/details.aspx?FamilyId=0BAF2B35-C656-4969-ACE8-E4C0C0716ADB&displaylang=en

or so.

After that, there was still something wrong with the PATH setup for cl.  I edited my

/media/win/c/Program Files/Microsoft Visual Studio 8/Common7/Tools/vsvars32.bat

file so that the INCLUDE variable includes the correct path to the SDK include files, something like:

@set PATH=C:\Program Files\Microsoft Visual Studio 8\Common7\IDE;C:\Program Files\Microsoft Visual Studio 8\VC\BIN;C:\Program Files\Microsoft Visual Studio 8\Common7\Tools;C:\Program Files\Microsoft Visual Studio 8\SDK\v2.0\bin;C:\WINDOWS\Microsoft.NET\Framework\v2.0.50727;C:\Program Files\Microsoft Visual Studio 8\VC\VCPackages;C:\Program Files\Microsoft Platform SDK\Bin;%PATH%
@set INCLUDE=C:\Program Files\Microsoft Visual Studio 8\VC\INCLUDE;C:\Program Files\Microsoft Platform SDK\Include;%INCLUDE%
@set LIB=C:\Program Files\Microsoft Visual Studio 8\VC\LIB;C:\Program Files\Microsoft Visual Studio 8\SDK\v2.0\lib;C:\Program Files\Microsoft Platform SDK\Lib%LIB%
@set LIBPATH=C:\WINDOWS\Microsoft.NET\Framework\v2.0.50727

There are a number of discussions about this on the web, e.g.,

http://www.gamedev.net/community/forums/topic.asp?topic_id=440340

I hope this helps,

%%%%%%

Andreas bug report:

<!DOCTYPE html
    PUBLIC "-//W3C//DTD XHTML 1.0 Strict//EN"
    "http://www.w3.org/TR/xhtml1/DTD/xhtml1-strict.dtd">
<html xmlns="http://www.w3.org/1999/xhtml" lang="en" xml:lang="en">
<head>
 <title>#3 (Problem compiling on AIX (CppAD trouble?)) - Optimization Services - Trac</title><link rel="start" href="/OS/wiki" /><link rel="search" href="/OS/search" /><link rel="help" href="/OS/wiki/TracGuide" /><link rel="stylesheet" href="/OS/chrome/common/css/trac.css" type="text/css" /><link rel="stylesheet" href="/OS/chrome/common/css/ticket.css" type="text/css" /><link rel="icon" href="/OS/chrome/common/trac.ico" type="image/x-icon" /><link rel="shortcut icon" href="/OS/chrome/common/trac.ico" type="image/x-icon" /><link rel="alternate" href="/OS/ticket/3?format=rss" title="RSS Feed" type="trac.ticket.Ticket" /><link rel="alternate" href="/OS/ticket/3?format=tab" title="Tab-delimited Text" type="trac.ticket.Ticket" /><link rel="alternate" href="/OS/ticket/3?format=csv" title="Comma-delimited Text" type="trac.ticket.Ticket" /><style type="text/css">
</style>
 <script type="text/javascript" src="/OS/chrome/common/js/trac.js"></script>
</head>
<body>


<div id="banner">

<div id="header"><a id="logo" href="http://www.coin-or.org/"><img src="/OS/chrome/common/coin_banner.jpg" width="80" height="80" alt="" /></a><hr /></div>

<form id="search" action="/OS/search" method="get">
 <div>
  <label for="proj-search">Search:</label>
  <input type="text" id="proj-search" name="q" size="10" accesskey="f" value="" />
  <input type="submit" value="Search" />
  <input type="hidden" name="wiki" value="on" />
  <input type="hidden" name="changeset" value="on" />
  <input type="hidden" name="ticket" value="on" />

 </div>
</form>



<div id="metanav" class="nav"><ul><li class="first"><a href="/OS/login">Login</a></li><li><a href="/OS/settings">Settings</a></li><li><a accesskey="6" href="/OS/wiki/TracGuide">Help/Guide</a></li><li><a href="/OS/about">About Trac</a></li><li class="last"><a href="/OS/register">Register</a></li></ul></div>
</div>

<div id="mainnav" class="nav"><ul><li class="first"><a accesskey="1" href="/OS/wiki">Wiki</a></li><li><a accesskey="2" href="/OS/timeline">Timeline</a></li><li><a accesskey="3" href="/OS/roadmap">Roadmap</a></li><li><a href="/OS/browser">Browse Source</a></li><li class="active"><a href="/OS/report">View Tickets</a></li><li class="last"><a accesskey="4" href="/OS/search">Search</a></li></ul></div>

<div id="main">




<div id="ctxtnav" class="nav">
 <h2>Ticket Navigation</h2>
</div>

<div id="content" class="ticket">

 <h1>Ticket #3 <span class="status">(new defect)</span></h1>

<div id="searchable">
<div id="ticket">
 <div class="date">
  <p title="08/10/07 12:53:44">Opened 1 week ago</p>
 </div>
 <h2 class="summary">Problem compiling on AIX (CppAD trouble?)</h2>
 <table class="properties">
  <tr>

   <th id="h_reporter">Reported by:</th>
   <td headers="h_reporter">andreasw</td>
   <th id="h_owner">Assigned to:</th>
   <td headers="h_owner">somebody</td>
  </tr><tr>
    <th id="h_priority">Priority:</th>

    <td headers="h_priority">major</td>
    <th id="h_milestone">Milestone:</th>
    <td headers="h_milestone"></td></tr><tr>
    <th id="h_component">Component:</th>
    <td headers="h_component">component1</td>
    <th id="h_version">Version:</th>

    <td headers="h_version"></td></tr><tr>
    <th id="h_keywords">Keywords:</th>
    <td headers="h_keywords"></td>
    <th id="h_cc">Cc:</th>
    <td headers="h_cc"></td></tr><tr></tr>
 </table>
  <form method="get" action="/OS/ticket/3#comment" class="printableform">
   <div class="description">

    <h3 id="comment:description">
     Description
    </h3>
    <p>
I'm trying to compile on AIX (IBM's xlC compiler).
</p>
<p>
The first set of problems comes in because CppAD is using the identifiers "isfinite" and "isnan", 
which on AIX is the name of preprocessor macros, defined in /usr/include/math.h.  I got around this 
problem buy renaming "isfinite" and "isnan" in CppAD's near_equal.hpp and nan.hpp.  But maybe the 
issue is related that you use &lt;math.h&gt; in C++ code, whereas the C++ standard says you should 
use &lt;cmath&gt;.  In Ipopt, I check for each C header if the C++ version is there, and if not, 
if the C version is there, and include the files accordingly.  There is macro for testing in coin.m4.  
In Ipopt's configure.ac I use:
</p>

<pre class="wiki">AC_COIN_CHECK_CXX_CHEADER(math)
</pre><p>
and then the source code has:
</p>
<pre class="wiki">#ifdef HAVE_CMATH
# include &lt;cmath&gt;
#else
# ifdef HAVE_MATH_H
#  include &lt;math.h&gt;
# else
#  error "don't have header file for math"
# endif
#endif
</pre><p>
Once I'm able to get through that point, I get tons of error messages like the following:
</p>
<pre class="wiki">"/u/andreasw/home4/COIN-svn/CoinAll/branches/all-trunk/cppad/../cppad/local/std_math_unary.hpp", line 322.9: 1540-0215 (S) The wrong number of arguments have been specified for "CppAD::AD&lt;double&gt;::cos() const".
"/u/andreasw/home4/COIN-svn/CoinAll/branches/all-trunk/cppad/../cppad/local/std_math_unary.hpp", line 322.9: 1540-0700 (I) The previous message was produced while processing "CppAD::AD&lt;double&gt;::cos() const".
"/u/andreasw/home4/COIN-svn/CoinAll/branches/all-trunk/cppad/../cppad/local/std_math_unary.hpp", line 322.9: 1540-0700 (I) The previous message was produced while processing "CppAD::cos&lt;double&gt;(const AD&lt;double&gt; &amp;)".
"../../../../../../../CoinAll/branches/all-trunk/OS/src/OSCommonInterfaces/OSnLNode.cpp", line 1157.23: 1540-0700 (I) The previous message was produced while processing "OSnLNodeCos::constructCppADTape(std::map&lt;int,int,std::less&lt;int&gt;,std::allocator&lt;std::pair&lt;const int,int&gt; &gt; &gt; *, CppAD::vector&lt;CppAD::AD&lt;double&gt; &gt; *)".

</pre>
   </div>
  </form>
</div>









 </div>

 <script type="text/javascript">
  addHeadingLinks(document.getElementById("searchable"), "Permalink to $id");
 </script>
</div>
<script type="text/javascript">searchHighlight()</script>
<div id="altlinks"><h3>Download in other formats:</h3><ul><li class="first"><a href="/OS/ticket/3?format=rss">RSS Feed</a></li><li><a href="/OS/ticket/3?format=tab">Tab-delimited Text</a></li><li class="last"><a href="/OS/ticket/3?format=csv">Comma-delimited Text</a></li></ul></div>

</div>

<div id="footer">
 <hr />

 <a id="tracpowered" href="http://trac.edgewall.org/"><img src="/OS/chrome/common/trac_logo_mini.png" height="30" width="107"
   alt="Trac Powered"/></a>
 <p class="left">
  Powered by <a href="/OS/about"><strong>Trac 0.10.4</strong></a><br />
  By <a href="http://www.edgewall.org/">Edgewall Software</a>.
 </p>
 <p class="right">
  Visit the Trac open source project at<br /><a href="http://trac.edgewall.org/">http://trac.edgewall.org/</a>

 </p>
</div>



 </body>
</html>

for wget see Christopher G.  Lewis Windows wget

%%%%%%%%%%%%%%%%%%%

What we need for Ipopt

Hi JP,

Actually, I just saw now that you sent me also the output of configure.
And that one fails because you don't have a Fortran compiler, which at the
moment I assume is present for Ipopt, since it is required for almost all
possible configurations.

I could take the dependency out for a Fortran compiler, but that doesn't
solve the problem, since essentially any of the sparse linear solvers need
at least the Fortran runtime libraries.  For Ipopt's configure script to
work you definitely need:

1.  BLAS (either you have it installed already on your system (e.g.,
libblas on Linux) or you have the source code)
2. one of: ma27, MA57, Pardiso, WSMP, MUMPS (only for MUMPS there is a
get.Mumps script)

What are you guys trying to accomplish?  Some automated procedure to see if
trunk builds?  If so, why wouldn't you want to provide all dependencies
(also ASL and LAPACK) to make sure that more configurations of the code
work?

Thanks

%%%%%%%%%%%%%%



lindo




I have the Mac Intel Lindo API working. A bit of a kludge, but I created a directory on my machine

/opt/intel/cc/9.1.037/lib

Then I copied libimf.dylib and libirc.dylib into this directory.  
I built the OS project using the GNU build tools with



%%%%%%%%%%%%%%

Ted:

Kipp Martin wrote:
> Hi Ted:
>
>>
>> By the way, I would suggest you make sure that configuration fails for OS whenever cppad is not present, 
> since it appears that OS will not build without it (right?). I tried to configure it without cppad and it
>
> The more I think about the above, the less I understand it. True, OS will not build without CppAD but CppAD 
> is in the Externals file. Same is true for CoinUtils. OS will not build without CoinUtils, but it is in the 
> Externals file so I don't check for it.  Why should I treat CppAD different than any other COIN-OR project?

Actually, I'm pretty sure that none of the projects are actually doing this 100% right. 
What I think should happen is that for required external projects, the configure script 
should check for their existence and fail if they are not present. For optional projects, 
your code should use the symbols defined for you by autoconf to make sure that the code 
compiles properly when the optional module is not present, i.e., there should be blocks like

#ifdef COIN_HAS_XXX

#endif

that are skipped whenever XXX is not present. In your configure.ac, the line

AC_COIN_MAIN_SUBDIRS(CoinUtils)

does actually check for the existence of other COIN projects, but it is up to you to figure out what to do 
if something is not present. Take a look at the configure.ac for Ipopt to see how this works.

Ideally, your code would never fail to compile because something is not present. Either the configuration 
should fail or the code should be able to deal with the lack of presence of some module. So you actually 
*do* check for the presence of other COIN projects. However, as far as I know, there is no implemented test for cppad, 
presumably since it does not use the autotools like the other projects. Currently, if you add

AC_COIN_MAIN_SUBDIRS(CoinUtils)

it doesn't correctly detect its presence. You have this line in your configure.ac, but if you check the logs, 
it probably says that cppad is not present. I think this is probably because it does not have its own configure script. 
In that sense, it probably needs to be treated like third-party source code.

The reason for making sure that your configuration fails when something is not present, even though it is in 
your externals, is because your externals aren't used when other projects pull in your project (as I am doing 
with CoinAll). I don't really have any way of knowing which things  in your externals are required for me to build 
your project with the default options. Hence, I didn't include cppad at first. When I could not get Ipopt to build, 
I actually tried to build CoinAll without Ipopt and again, OS configured just fine, but failed to build because Ipopt 
was not present. By the way, is it really true that you need Ipopt to use OS? What if I am only interested in using 
Clp through OS and don't care about Ipopt? Shouldn't I be able to build it without Ipopt? I'm guessing that this is 
possible, but it doesn't happen automatically whenever Ipopt is not present, as it should.

So the bottom line is that all the COIN projects *should* actually be treated that same way as I've described cppad 
should be treated. However, it's easier to do this for the projects that use autoconf.

I hope this makes sense. Andreas can correct me if I've misspoken anywhere :) .

Cheers,

Ted
--
Dr. Ted Ralphs
Associate Professor
Industrial and Systems Engineering
Lehigh University
(610)758-4784
ted at lehigh dot edu
www.lehigh.edu/~tkr2



%%%%%%%%%

JP




For OS the command looks like:
   vcbuild /u F:\nbBuildDir\OS\trunk\OS\MSVisualStudio\v8\OS.sln $ALL
Before running vcbuild a few environement variables need to set.
This is how I do that:
    "E:\Microsoft Visual Studio 8\Common7\Tools\vsvars32.bat"
    set LIB=E:\Microsoft Platform SDK for Windows Server 2003 R2\Lib
    set LIB=E:\Microsoft Visual Studio 8\VC\lib;%LIB%
    set INCLUDE=E:\Microsoft Platform SDK for Windows Server 2003 R2
\Include;%INCLUDE%



%%%%%%%%%



AC_COIN_HAS_PROJECT(cppad)
case $coin_has_cppad in
  unavailable | skipping)
    AC_MSG_ERROR([cannot find CppAD])
esac

Sorry, I didn't read your message well.

You are right, AC_COIN_MAIN_SUBDIRS(cppad) tells us that the cppad project is not available, since there is no 
configure script - and therefore the base directory configure script can't and shouldn't recurse into the cppad 
subdirectory.  There is actually no need for "AC_COIN_MAIN_SUBDIRS(cppad)" to appear in the configure.ac of OS' 
base directory.

However, the AC_COIN_HAS_PROJECT(cppad) test is independent from that, and that seems to do that right thing 
in OS (that was what I meant in my other messages).

Sorry for the confusion.

Andreas

%%%%%%
%%%%%%

Just a follow-up to one of Stefan's comments:

> > From: Stefan Vigerske [mailto:stefan@math.hu-berlin.de]
> > Sent: Wednesday, November 21, 2007 4:57 AM
> > To: Steven Dirkse
> > Cc: Kipp Martin; Jun Ma @ NWU; Robert Fourer; huanyuan sheng
> > Subject: Re: GAMS and OS
> >
> > - have the possiblity to link the GAMS I/O libs directly to an
> > OSiL-compatible solver by giving it an OSInstance object instead of
> > pointing it to an OSiL file. Not only that additional rounding errors by
> > writing an ASCII-representation and reading it again are avoided, also
> > it might be easier to implement advanced features like support for GAMS
> > BCH and for GAMS external functions (provided this is supported by OS).

If writing an ASCII representation and reading it again are done properly,
then there need not be any rounding errors.  The binary representation after
reading the ASCII can be guaranteed to be identical to the binary
representation before writing the ASCII.  This guarantee cannot be achieved
if the ASCII representation is limited to 12 characters, however, as in the
classical version of MPS form.

For more on this subject see Dave Gay's discussion at
www.ampl.com/REFS/rounding.pdf.  There also is a distribution of the
rounding routines that is open source -- see the initial comments to
www.netlib.org/ampl/solvers/dtoa.c -- but I don't know how this fits with
other licenses such as the CPL and GPL.


%%%%%%%%%%%%%%%%%%%

Gus and msys

> Bob -- I am ccing you on all this discussion about parsers since when you implement the David Gay stuff the 
reading of the numbers into text is done in parseosil.y and many of Gus' questions are relevant.  
I think he also has msys and through a fair amount of pain has installed flex and bison so perhaps you can 
leverage off of him.

Hi Bob,

msys _is_ fairly easy to install, but you have to know what you need,
and the website is very unhelpful. If you want flex and bison, you are
going to have to download from

http://sourceforge.net/project/showfiles.php?group_id=2345

the following files:

MSYS-1.0.11-20071204
bash-3.1-MSYS-1.0.11
bison-2.3-MSYS-1.0.11
flex-2.5.33-MSYS-1.0.11
regex-0.12-MSYS-1.0.11

The last one contains an important DLL, msys-regex-0.dll, without which flex
will not start. Unfortunately there is no documentation anywhere, and I was
banging my head against a wall for at least two days on this point.

You might want to get other files, such as

coreutils-5.97-MSYS-1.0.11
make-3.81-MSYS-1.0.11

but I don't think they are essential.

I still have not figured out the path thing, but I figure, I can write a batch
file that copies the files back and forth.


%%%%%%%%%%%%%%%%%

This is good. You might want to mention in the MSVS section that the flex and
bison available for windows do not allow the options to build a reentrant
parser (which we have to have for (at least) the <nonlinearExpressionTree>).
You could then point anyone interested in modifying the parsers to the entry in
section 4.2.4. (These six lines actually deserve their own number, but I don't
know where you stand on the nesting level. My suggestion would be to call it
"4.2.5 flex and bison".)

I also noticed another typo in the top third of page 39: The bison version
number should be 2.3, not 3.2.


%%%%%%%%%
%%%%%%%%%%%%%%%%%%%


Bob --

Right -- I am planning on using the functions strtod and dtoa described in
www.ampl.com/REFS/abstracts.html#rounding and available from netlib.

 #ifdef KR_headers
02644     (d, mode, ndigits, decpt, sign, rve)
02645     double d; int mode, ndigits, *decpt, *sign; char **rve;
02646 #else
02647     (double d, int mode, int ndigits, int *decpt, int *sign, char **rve)
02648 #endif
02649 
02650  /* Arguments ndigits, decpt, sign are similar to those
02651     of ecvt and fcvt; trailing zeros are suppressed from
02652     the returned string.  If not null, *rve is set to point
02653     to the end of the return value.  If d is +-Infinity or NaN,
02654     then *decpt is set to 9999.
02655
02656     mode:
02657         0 ==> shortest string that yields d when read in
02658             and rounded to nearest.
02659         1 ==> like 0, but with Steele & White stopping rule;
02660             e.g. with IEEE P754 arithmetic , mode 0 gives
02661             1e23 whereas mode 1 gives 9.999999999999999e22.
02662         2 ==> max(1,ndigits) significant digits.  This gives a
02663             return value similar to that of ecvt, except
02664             that trailing zeros are suppressed.
02665         3 ==> through ndigits past the decimal point.  This
02666             gives a return value similar to that from fcvt,
02667             except that trailing zeros are suppressed, and
02668             ndigits can be negative.
02669         4,5 ==> similar to 2 and 3, respectively, but (in
02670             round-nearest mode) with the tests of mode 0 to
02671             possibly return a shorter string that rounds to d.
02672             With IEEE arithmetic and compilation with
02673             -DHonor_FLT_ROUNDS, modes 4 and 5 behave the same
02674             as modes 2 and 3 when FLT_ROUNDS != 1.
02675         6-9 ==> Debugging modes similar to mode - 4:  don't try
02676             fast floating-point estimate (if applicable).
02677
02678         Values of mode other than 0-9 are treated as mode 0.
02679
02680         Sufficient space is allocated to the return value
02681         to hold the suppressed trailing zeros.
02682     */



sample code

static UString integer_part_noexp(double d)
{
    int decimalPoint;
    int sign;
    char *result = kjs_dtoa(d, 0, 0, &decimalPoint, &sign, NULL);
    int length = strlen(result);

    UString str = sign ? "-" : "";
    if (decimalPoint == 9999) {
        str += UString(result);
    } else if (decimalPoint <= 0) {
        str += UString("0");
    } else {
        char *buf;

        if (length <= decimalPoint) {
            buf = (char*)malloc(decimalPoint+1);
            strcpy(buf,result);
            memset(buf+length,'0',decimalPoint-length);
        } else {
            buf = (char*)malloc(decimalPoint+1);
            strncpy(buf,result,decimalPoint);
        }

        buf[decimalPoint] = '\0';
        str += UString(buf);
        free(buf);
    }

    kjs_freedtoa(result);

    return str;
}

See:

1) http://www.krugle.org/examples/p-UkvJ53OlMMjGJ0QO/number_object.cpp

2) http://www.krugle.org/examples/p-UkvJ53OlMMjGJ0QO/dtoa.h



I got a reply from Dave Gay about the lossless conversion issues.  All of the
relevant routines are in www.netlib.org/fp -- it's not necessary to search
through the ASL routines for them.  In particular, to get the shortest decimal
string that correctly represents a binary value, we can use

     g_fmt(register char *b, double x)

which is in www.netlib.org/fp/g_fmt.c.  This routine calls dtoa and converts
the return value and arguments to the appropriate string.  In the process it
inserts a sign, decimal point, and exponent if appropriate.

As you suspected, dtoa.c contains its own strtod routine because, at the time
Dave wrote dtoa, many strtod routines in other libraries did not do the
conversion in a lossless way.  Dave considers it likely that many stdlib
implementations get this right by now, but I guess there is still no easy way
to be sure that they all get it right.

A look at the change log suggests that some people are actively using these
routines independently of ASL.


%%%%%%%%%%%%%%%%%%%%%%


cygwin gfortran

GMP
libgmp3  GMP librarry

MPFR

libmpfr1

%%%%%%%%%%%%%%%%%%%%%%%

The patch for mumps


Hi Kipp and Andreas,

To try and bring a conclusion to the saga of building Ipopt with Mumps in Msys, 
here is a patch file for the changes I had to make to Mumps to get it to compile 
with gfortran 4.2 in Msys. I've included a version of the get.Mumps script that 
will automatically download the source and apply the necessary patch. Andreas, 
it seems this is the easiest of the solutions we discussed and it seems to work fine. 
Do you think we can just check in the patch and the new version of the script?

Kipp, if you want to apply the patch to already downloaded code, please execute

patch -p0 < mumps.gcc.patch

in the ThirdParty/Mumps directory. I'm a little confused as to why you seem to be 
having different compilation issues than I did, but try this patch and see if it works.

Cheers,

Ted -- email of 12/10/2007

%%%%%%
Using SYMPHONY remotely.

http://calvin.ie.lehigh.edu/os/OSSolverService.jws



serviceLocation http://calvin.ie.lehigh.edu/os/OSSolverService.jws
osil ../data/osilFiles/p0201.osil
solver symphony
osol ../data/osolFiles/symphony.osol
osrl ./test.osrl
browser /Applications/Firefox.app/Contents/MacOS/firefox

<?xml version="1.0" encoding="UTF-8"?>
<osol xmlns="os.optimizationservices.org"
      xmlns:xsi="http://www.w3.org/2001/XMLSchema-instance"
      xsi:schemaLocation="os.optimizationservices.org
      http://www.optimizationservices.org/schemas/2.0/OSiL.xsd">
  <general>

  </general>
    <optimization>
    	<other name="num_proc">4</other>
    </optimization>
</osol>



%%%%%%%%%%%%%%%%%%%%%%%%%%%%%%%%
I can reproduce what you say on a Mac that should be similar to yours.

The problem might be that the use of atof triggers some SL routines
that ask for _Stderr.
However, Stderr gets defined in stderr.c of the ASL library:
FILE *Stderr;

Also "nm amplsolver.a | grep Stderr" produces
00000010 C _Stderr

Maybe the "C" (=common) does not count as symbol definition? I do not
really understand what "man nm" tells me about this.



%%%%%%%%%%%%%%%%%%%%%%%%%%%%%%%%%%


I've run Ipopt's configure from the vpath-directory.
To be more elaborate:
I've two directories:
Ipopt-trunk is where I have checked out Ipopt/trunk.
Ipopt-shared is the vpath-directory where I build Ipopt (using shared libs).
What to do is to:
1. Put the HSL source into Ipopt-trunk/ThirdParty/HSL
2. In Ipopt-shared, call
      ../Ipopt-trunk/configure --enable-loadable-library
3. Patch Ipopt-shared/libtool
4. In Ipopt-shared (or in its ThirdParty/HSL subdir) call make.

Now the gfortran call that builds the libhsl.dylib should next to the -dynamiclib 
argument also have a -single_module argument. If that is not sufficient, 
then one need to add also ADD_FFLAGS="-fno-common" to the configure call in step 2. 
I have not figures this out.

It should also be possible to run only the configure in ThirdParty/HSL, but then one 
need to give a lot of options for prefix or subdir... (see beginning of ThirdParty/HSL/config.log).

%%%%%%%%%%%%%%%%%%%%%%%%%%%%

Code for user defined variables.

See:

http://www.gerad.ca/~orban/drampl/def-vars.html



 k = (expr_v *)e - VAR_E;
 if( k >= n_var ) {

     // This is a common expression. Find pointer to its root.

     j = k - n_var;
     if( j < ncom0 )
         com_expr = CEXPS;
     else
         com_expr = CEXPS1 - ncom0;

     Printf( "    Nonlinear part:\n" );
     display_expr( (com_expr + j)->e, asl );

     nlin = (com_expr + j)->nlin; // Number of linear terms
     if( nlin > 0 ) {
         Printf( "\n    Linear terms:\n" );
         L = (com_expr + j)->L;
         for( i = 0; i < nlin; i++ ) {
             vp = (expr_v *)((char *)L->v.rp - ((char *)&ev.v - (char *)&ev));
             Printf( " %-g x[%-d]", L->fac, (int)(vp - VAR_E) );
             L++;
         }
     }
 }
%%%%%%%%%%%%%%%%%%%%%%%%%%%
%%%%%%%%%%%%%%%%%%%%%%%%%%%

> > >
> > > These are often called "defined variables" in descriptions of AMPL.  An nl
> file
> > > gives statistics for the number of defined variables appearing
> > >
> > >      b   in both the objective and constraints
> > >      c   in two or more constraints but not any objectives
> > >      o   in two or more objectives but not any constraints
> > >      c1  in only one constraint and no objectives
> > >      o1  in only one objective and no constraints
>
> I am curious, where would the user find the above information? What is
> particularly confusing is that three lines above
>
>   0 0 0 3 0	# common exprs: b,c,o,c1,o1
>
> is the line
>
> 0 0 0 0 0	# discrete variables: binary, integer, nonlinear (b,c,o)
>
>
> so the triple (b, c, o) has two distinct meanings.


%%%%%%%%%%%%%%%%%%%%%%%%%%%

Yes.>
I believe this is possible.

> >is it possible to tell vcbuild to skip certain configurations
> >rather than running them all?

NBbuildConfig.py has the line:
    vcbuild='vcbuild /u ' + slnFileName + ' $ALL'

I'm pretty sure that the $ALL means build all configurations.
It could be changed to be the name of the configuration to be built.

Here is some documentation I just found:
http://msdn2.microsoft.com/en-us/library/kdxzbw9t.aspx

JP Fasano
STSM, Watson Math Department
jpfasano@us.ibm.com
(914)945-1324  (tie line 862-1324)


%%%%%%%%%%%%%%%%%%%
Gus and Visual Studio

Quoting Kipp Martin <Kipp.Martin@ChicagoGSB.edu>:

> Hi Gus:
>>
>> Any thoughts?
>
> That seems to have worked in terms of VS recognizing that the project is there. 
> However, when I now open OS.sln in v9 inside Visual Studio it says: "the solution or 
> project you are opening was created in a different version of Visual Studio ..." 
> Then it wants to go through a conversion process.  So something is now wrong with the solution file in  v9.

I don't have v9 running on this machine, but the conversion process is trivial.
If you do not want to let MSVS do the conversion for you, just edit the file
(it's XML) and change the version number to 10.00 and the package title to
Visual Studio 2008. I go in the opposite direction, changing the version to
9.00 and the title year to 2005.

> Was the osRemoteTest project you committed created in v8? Are you able to open 
> OS.sln in v9 from Visual Studio without it asking to convert?
>
> Also, how do you turn on/off a project so that is is/is not built from vcbuild?  
> I don't see how to do that so I can't test the Windows Popup blocker issue.  
> I am pretty stumped, especially since the WindowsErrorPopupBlocker(); code is in 
> unitTest and unitTest works from vcbuild.

Again there are two ways to do it. In MSVS you select the configuration you
want, and then select Configuration Manager from the build menu. Just click on
the projects you want to build. You have to do this separately in each
configuration. (Just be careful _not_ to change the configuration in the
configuration manager. It's an easy tab, so it is very tempting, but it's done
funny things to my setup.)

You can also select projects in an editor. If you open OS.sln you will see at
the end of the file a bunch of lines that have project numbers and blah blah
blah Activecfg = ...
These lines tell MSVS which projects are included in which configurations. There
are also lines with ...Build.0 = ... These lines tell MSVS which projects
should be built by default. If you want to change that, you can simply add or
delete the appropriate entries.

Hope that helps

gus


> Hi Gus:
>
> Okay, for v9 I went into Configuration Manager (the debug tab was selected) 
> and I turned on fileupload and all of the examples including your new osRemoteTest. 
> Then I saved the OS.sln file. Then went to the command line and ran vcbuild OS.sln. 
> All of the projects built with no problem!
>
> I think you said not to change the tab in the Configuration Manager so how do 
> I turn these guys on in both release and debug mode?

You should have a list box on the tool bar that shows the currently active
configuration. If you select Release from there and then go back into the
Configuration Manager, everything is fine. The problem comes about when you
change the configuration inside the Configuration Manager, as MSVS then assumes
that in addition to the selections you make, you would like the project in the
configuration you change to be built using the active configuration. (This
sounds very confusing,  so I'll give an example.)

Suppose you have Debug as your active configuration and open the Configuration
Manager. If you then click Release inside the Configuration Manager and turn
on, say, osTest, you tell MSVS to build Release with the Debug information
turned on, that is, osTest --- and possibly all the other projects, can't
remember for sure --- will have entries in the .sln files changed from

{project number}.Release.ActiveCfg = Release|Win32 (or something close to that)

to

{project number}.Release.ActiveCfg = Debug|Win32 (which presumably you didn't
want)

But if you close the Configuration Manager before you change the configuration,
everything is fine.

Hope this helps

gus


%%%%%%%%%%%%%%%

New improvement from stefan

Hi,

I just managed to build the CoinAll system (from BSP) on a Windows system 
with cl, f2c, no fortran 90 compiler, and user given mumps libraries without 
having to patch the Ipopt configure scripts.
I have put all files that I used at

http://www.gams.com/~svigerske/mumps/

(Kipp, this are essentially the same as you used before for the same thing, 
just reduced to the essential ones.)

The readme.txt just says that i used the following site script:

with_mumps_lib="c:/cygwin/home/stefan/mumps/libcoinmumps.lib \
c:/cygwin/home/stefan/mumps/blas_ifort.lib \
c:/cygwin/home/stefan/mumps/intel-libs/libmmt.lib \
c:/cygwin/home/stefan/mumps/intel-libs/libirc.lib \
c:/cygwin/home/stefan/mumps/intel-libs/svml_disp.lib \
c:/cygwin/home/stefan/mumps/intel-libs/ifconsol.lib \
c:/cygwin/home/stefan/mumps/intel-libs/libifcoremt.lib \
c:/cygwin/home/stefan/mumps/intel-libs/libifport.lib"

with_mumps_incdir="c:/cygwin/home/stefan/mumps/inc"

It's not nice yet, but an improvement I think ;-) .

Stefan

--



Ted:

Mac OS X issues

I did some more digging and for those who are interested, I seem to have
gotten to the bottom of how to successfully build CoinAll on OSX 10.5
(Leopard). As I had suspected, the problem is essentially a
name-mangling issue. The bottom line is that between 10.4 and 10.5,
Apple changed the implementation of a lot of the system routines in
order to gain UNIX certification (who knew there still was such a
thing?). In order to maintain backwards compatibility, however, the
symbols associated with the new versions of system calls have $UNIX2003
appended to the symbol name in libraries compiled under 10.5. For more
details, see here:

http://developer.apple.com/releasenotes/Darwin/SymbolVariantsRelNotes/index.html

The problem arises because ASL itself reimplements one of the Unix
system calls that was also reimplemented by OSX (strtod). Presumably to
avoid name conflicts with the original system call, the line

#define strtod strtod_ASL

was inserted into the file dtoa1.c, so that the system call would be
replaced with the ASL reimplementation everywhere. The compiler then
apparently gets a little confused and thinks that the symbol strtod_ASL
refers to an OSX system call and helpfully appends $UNIX2003 to its
symbol name strtod_ASL in the amplsolver.a library. The linker, however,
does not then seem to properly link calls to strtod_ASL in other object
files to this new definition. Got that? It's a little confusing and I
think it's actually a bug in the compiler.

There are a number of possible fixes, however. The easiest one seems to
be to use a compiler option that forces the use of the old system calls
in order to allow building of codes on 10.5 that run on older variants:

?mmacosx-version-min=10.4

This is also the same as defining "MAC_OSX_DEPLOYMENT_TARGET=10.4".
Since we probably want our binaries to be compatible with older versions
of OSX anyway and this will allow ASL to build properly, I would suggest
that we add that compiler option automatically for OSX in coin.m4. I'm
not sure whether older compilers will understand it, so if not, we'll
need to detect whether we are working version 10.5 or an earlier one.
Alternatively, we could simply modify the compile_unix_ASL script or
patch ASL itself. Thoughts?


%%%%%%%%%%%%%%%

How Osi works


Folks,

	The way it works for Osi is this:  The code for all OsiXXX solver
interface layers is always included in the Osi distribution.  If solver XXX is
not available, OsiXXX is not configured, built, or tested.  The configure script
sorts this out.

	For Osi/stable and Osi/releases, the default Externals specifies clp,
dylp, vol, and ThirdParty/Glpk (but unless the user downloads glpk, OsiGlpk will
not be enabled).

	For Osi/trunk, the default Externals adds Cbc and SYMPHONY, plus Cgl as
a dependency.  OsiCbc and OsiSym will be enabled.  There's a configure hook to
specify the underlying OsiXXX that's used by OsiCbc.  OsiClp is generally safe
and is the default.  OsiDylp may work, or may be broken due to incompatibilities
in libCbc.  Other solvers haven't been tested, to my knowledge.

	There's ongoing debate over the appropriateness of this set of
Externals, and ongoing debate over the viability (design-wise) of OsiCbc.

	Commercial solvers always require the user to specify the location of
the libraries and includes.  If the user gives a location for XXX, the
corresponding OsiXXX is configured and built.

%%%%%%%%%%%%%%

For GAMSlinks this may be necessary:


Seem to be the same thing we had a month ago.
Can you try if adding the following to GamsOS.cpp helps again?

extern "C" {
  double slvminf = 1;
  unsigned char G2DMATHNEW_exceptmsg[256] = "hack";
}


I've committed these lines into the repository now too, but you have to activate 
them by adding a -DUSE_UNUSED_SYMBOLS (I know, I'm very bad in naming) to the 
CXXFLAGS, see https://projects.coin-or.org/GAMSlinks/changeset/540
I am not convinced yet that this is not some bug in the compiler you use.

%%%%%%%%%%%%%%%%%%%%%%%

From Laci

OK, I have reverted the changes. Go ahead and create the releases.

--Laci

PS: btw, the way to revert the changes is really simple:
  svn checkout https://projects.coin-or.org/svn/OS/stable/1.1
  cd 1.1
  svn merge -r2088:2087 https://projects.coin-or.org/svn/OS/stable/1.1
  svn commit
Once you know it it's really logical.


%%%%%%%%%%%%%%

From Ted Ralphs 


You shouldn't need to do all this manually. Compilers pretty much all
define their own symbols automatically to allow you to detect when
they are being used and I believe the autotools or our own m4 scripts
define symbols to detect the OS. All did for the unistd.h problem in
other places (this reference was added by a developer who was unaware
of the proper fix) was add this:

#if !defined (_MSC_VER)
#include <unistd.h>            /* this defines sleep() */
#endif

That symbol is automatically defined when cl is the compiler. I'm not
sure of all the symbols off the top of my head, but __DARWIN is
defined is OS X, for example, and __MNO_CYGWIN is defined if a MINGW
compiler is used. Here's another line from one of my header files:

#if !defined (_MSC_VER) && !defined (__DARWIN) && !defined (__MNO_CYGWIN)
#include <sys/resource.h>
#endif

In general, almost anything you want to detect should already have a
symbol automatically defined, as it's doubtfl you are the first to
want this functionality  :) . So I think you can delete all that fancy
stuff from your configure.ac and just use the built-in symbols  :) 

%%%%%%%%%%%%%

%%%%%%%%%

Pietro -- OSOptions

> First question -- you
> mention LaTeX documentation on the Couenne Wiki but we cannot find the
> LaTeX documentation anywhere. Do you have a Couenne User's manual?

No, that's still in the works. The only documentation available is the 
doxygen one (that's what I mean in the wiki page), which becomes 
available with make doxygen. I hope to complete the manual after the 
end of this semester.

> In your src/main/BonCouenne.cpp we see you set, for example,
>
> bonmin.setDoubleParameter (...)
>
> were bonmin is a CouenneSetup object. Are you setting a Bonmin or a
> Couenne option here?

That's a Bonmin option.

> Are there separate Bonmin and Couenne options?

Yes. There are Couenne-specific options, defined in the 
registerOptions() methods of CouenneCutGenerator, CouenneProblem, and 
others, and that you can set using the Couenne option file or the 
CouenneSetup. The branch&bound general options (strong branching, max 
time, and others) are inherited from Bonmin. I believe you can also 
set the latter ones within a CouenneSetup.

> In our OSCouenneSolver we define
>
> Ipopt::SmartPtr<TMINLP> tminlp_;
>
> and then
>
> tminlp_ = new BonminProblem( osinstance, osoption, osresult);
>
> where our BonminProblem inherits from class TMINLP.
>
> Should we be setting the Bonmin options in one of the TMNLP methods? Or
> should we set all options (Couenne and Bonmin) through a CouenneSetup
> object?

A CouenneSetup object inherits from the analogous BonBabSetupBase 
object, therefore you can set all options through the CouenneSetup 
object. I hope Pierre can confirm or further clarify on that -- he 
wrote most of the BonCouenne*.?pp code.

%%%%%%%%%%%%%%%%%
Using the gnu debugger

When I want to debug a program p, then I do
$ gdb p
and in gdb I say "run".

For example:

$ gdb unitTest
GNU gdb 6.6.50.20070726-cvs
Copyright (C) 2007 Free Software Foundation, Inc.
GDB is free software, covered by the GNU General Public License, and you are
welcome to change it and/or distribute copies of it under certain
conditions.
Type "show copying" to see the conditions.
There is absolutely no warranty for GDB.  Type "show warranty" for details.
This GDB was configured as "i586-suse-linux"...
break Using host libthread_db library "/lib/libthread_db.so.1".
(gdb) break unitTest.cpp:263
Breakpoint 1 at 0x804e037: file
../../../GAMSlinks-trunk/OS/test/unitTest.cpp, line 263.
(gdb) run
Starting program: /home/stefan/work/coin/GAMSlinks-debug/OS/test/unitTest
[Thread debugging using libthread_db enabled]
[New Thread 0xb7bdbb70 (LWP 16655)]
START UNIT TEST
[Switching to Thread 0xb7bdbb70 (LWP 16655)]

Breakpoint 1, main (argC=1, argV=0xbff15f04) at
../../../GAMSlinks-trunk/OS/test/unitTest.cpp:263
263             int nOfTest = 0;
(gdb) bt
#0  main (argC=1, argV=0xbff15f04) at
../../../GAMSlinks-trunk/OS/test/unitTest.cpp:263


%%%%%%%%%%%%%%%%%%%%%%%%%%%%%

Jun oc Schema versioning



>> The strategy is quite clear. In general
>> 1. The version of the schema decides the version of our code.
>> If the version of our schema is 2.0, all the code is developed against 2.0 and upgrade with this major version. We change only the minor version.
>
> Here is the problem or confusion on my part.  When you say "minor version" above. it looks like you are referring to the second digit in the schema number. Is this correct? However, "minor version" for the C++ code is the third digit.
>> 2. If the schema either changes extremely significantly or becomes backward incompatible, we will discuss moving to the next major version.
>> In my opinion, it has to be REALLY significant to justify a change of our current 2.0 to 3.0.
>> Adding of the matrix/cone programming doesn't justify moving to 3.0. In fact adding of any extension only justifies to move the minor version, e.g. 2.0 to 2.1.
Sorry, should have been more clear on this. In my convention or in general software engineering practice:
a.b.c -> a is the major version, b is the minor version and c is the build version
I am saying "a" should be stable. "b" should be our milestone release.
From time to time, we have bug fixes and builds. That should be going to "c".
For the schema standard it probably should just be remaining at a.0.

> Right, once again your minor version does not mean the same thing as in the COIN-OR sense. So here is the issue:
>
> The latest release of OS on SVN is 2.0.1 where the .1 is the minor version. The minor version gets incremented when bug fixes are made to stable 2.0. So if we made another bug fix we would be up to 2.0.2. However, this next release of OS represents s LOT MORE than just bug fixes. But the API does not change. So we do not move to release 3.0 exactly as you say. I agree totally. But since we are do so much more than bug fixes, as per COIN-OR policy, we should not call this 2.0.2, we should call this 2.1.0.
Agreed.

> So now users will see OS release 2.1.0 but the schema version is only 2.0. I am worried that this will confuse users.
I am not that worried. Anybody can implement their code or a piece of code against our schema; not just us.
Even for us, we will implement java, and .net, possibly all versioned differently,but all against the same schema.
We just happen to have our convention to go line up our version with schema version.
It's the same thing as all other technologies:
Web Service (SOAP), xml, http, html, css hardly ever move a version. But all the implementation constantly upgrade.
Another example is Java specification,  right now we say Java 5 or Java 6 (we don't say Java 5.1 or 6.0.1)
While all the java sdk or run time implementation (e.g. sun jre) has all the minor versions.

I think users are fine with it. Even if they are a bit confused, they should be safe and cannot do anything wrong with it anyway.

>>
>> By our strategy,
>> 1. We shouldn't release code against 3.0, because there is no 3.0 schema.
>> 2. We probably shouldn't change the schema version, or at the most a minor version upgrade.
>> The matrix/cone stuff are not ready to be versioned. So there is nothing significant in our schema are to be versioned with our new code version.
>
> So you are saying use 2.0 for the schema and 2.1.0 for the C++  code. Is this the correct interpretation of your email?
Yes. as long as our schema extensions are backward compatible.
Moreover, in our current schema version, we have the safe mechanism of annotations on staging.
For example, an experimental element is in a 2.0 schema.
The fact the element is in a stage before versioned means we are free to change without being responsible to the users.
When they are versioned, they will be versioned with 2.0. So the staging is more like a buffer strategy to a stable versioning strategy.
We should always be responsible only to the existing users of versioned elements.
But as argued, they are safe by our strategy: we don't easily change versioned elements.
When an element is designed, we think of everything thing certain and uncertain (providing extension points) and prepared to extend them in a backward compatible way.

Maybe it helps to think in this way:
A simple and common versioning strategy is 2.0 alpha -> 2.0 beta -> 2.0 final on the entire software (coarse-grained).
but our schema versioning is more fine-grained (think of "tiered", "parallel at different pace", "structurally rich" etc.)
We have 2.0, final on element 1, beta on element 2, alpha on element 3, NOT on the entire schema!
We release 2.0 final, so that the common element 1 can be used by many users eagerly waiting there. (90% of the market).
We don't have the luxury of making element 2 and 3 final yet; that will take 2 more years,
but 90% of users will use only element 1 and don't care about element 2 and 3.

Now during the mean time element 2 will be worked on and we will try to move it from beta to 2.0 final as well.
Alpha and beta correspond to our many stages, only we have more.
The reason is that we are promoting a standard, not just a software.
A standard has to be stable.
You don't see xml being versioned from 1.0->2->2.5->3->4. If w3c does that, it will only destroy the adoption or market position of xml.
Of course, this requires an extremely well-thought design and process in the beginning so that we HAVE to think of EVERYTHING upfront.
SOAP is 1.2 (the only public release), and even though later they found some weakness, and even design flaw, they don't upgrade the version.
Changing a design drawbacks is far less important then maintaining the stability of a standard.
SOAP has many extension points, and the extension points start to be versioned separately, sometime even in a different standardization body.
For example, if we ever find some weakness in OS core later, our default action is to probably just to eat it or live with it, instead of changing it.

Our process of using fine-grained multi-tier "structured stages" versioning strategy should put as in a safer place.
Usually people (including us) do not have this luxury to the code versioning, it practically too much hassle. So we version the entire distribution instead of on each .cpp file.
But our modular extensible schema standard element design allows us to have such a luxury.

Jun





> Thanks
>
>
> -- 
> Kipp Martin
> Professor of Operations Research
> and Computing Technology
> Booth School of Business
> University of Chicago
> 5807 South Woodlawn Avenue
> Chicago, IL 60637
> 773-702-7456
> kmartin@chicagobooth.edu
> http://kipp.chicagobooth.edu
> http://projects.coin-or.org/OS
>
