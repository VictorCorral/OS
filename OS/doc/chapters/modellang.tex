\division{OS Support for Modeling Languages, Spreadsheets and Numerical Computing Software}\label{section:modellang}

Algebraic modeling languages can be used to generate model instances as input to an OS compliant solver.
We describe two such hook-ups, {\tt OSAmplClient} for AMPL\index{AMPL}, and {\tt CoinOS} for GAMS\index{GAMS} (version 23.8 and above).


\subdivision{AMPL Client:  Hooking AMPL to Solvers}\label{section:amplclient}

\index{OSAmplClient@{\tt OSAmplClient}|(}
\index{AMPL|(}




%This section is based on the assumption that the user has installed  AMPL  on his or her machine.   
It is possible to call all of the COIN-OR solvers listed in %Section~\ref{section:overview} 
Table~\ref{table:configurations}~(p.\pageref{table:configurations})
directly from the  AMPL (see {\tt http://www.ampl.com}) modeling language.  In this discussion we assume 
the user has already obtained and installed AMPL.
\ifdevelop  
Both the binary download described in Section~\ref{section:obtainingbinaries}
and the unix and Windows builds (Section \ref{section:unixbuilds}
and~\ref{section:windowsinstall}, respectively) contain
\else
The binary download described in Section~\ref{section:obtainingbinaries}
contains
\fi
%In  the download described in Section~\ref{section:binary} there is 
an executable, {\tt OSAmplClient.exe},
that is linked to all of the COIN-OR solvers  listed in Table~\ref{table:configurations}. %Section~\ref{section:overview}.   
From the  perspective of AMPL, the   {\tt OSAmplClient} acts like an AMPL ``solver''.    
The {\tt OSAmplClient.exe}   can be used to solve problems either locally or remotely.   


\subsubdivision{Using OSAmplClient for a Local Solver}\label{section:localampl}

In the following discussion we assume that the AMPL executable {\tt ampl.exe}, the {\tt OSAmplClient},  
and the test problem {\tt  eastborne.mod}\index{eastborne.mod@{\tt eastborne.mod}|(}
 are all in the same directory.  

The  problem instance {\tt eastborne.mod} is an AMPL model file included in the OS distribution 
in the {\tt amplFiles}\index{amplFiles@{\tt amplFiles}} directory.  To solve this problem locally 
by calling {\tt OSAmplClient.exe} from AMPL, first start AMPL and then open the {\tt eastborne.mod} file 
inside AMPL.  The test model {\tt eastborne.mod} is a linear integer program. 


%\begin{verbatim}
%# take in sample integer linear problem
%# assume the problem is in the AMPL directory
\begin{verbatim}
model eastborne.mod;
\end{verbatim}

The next step is to tell AMPL that the solver it is going to use is {\tt OSAmplClient.exe}. 
Do this by issuing the following command inside AMPL.

%\begin{verbatim}
%# tell AMPL that the solver is OSAmplClient
\begin{verbatim}
option solver OSAmplClient;
\end{verbatim}
\ifbible
%
This form of the command assumes that the {\tt OSAmplClient} executable is on the search path. If this is not the case, an explicit path to the executable can be given instead, for instance

\begin{verbatim}
option solver "./OSAmplClient";
\end{verbatim}
\fi

It is not necessary to provide the  {\tt OSAmplclient.exe} solver with any options. 
You can just issue the {\tt solve} command in AMPL as illustrated below.  

%\begin{verbatim}
%# solve the problem
\begin{verbatim}
solve;
\end{verbatim}

Of the six methods described in Section~\ref{section:ossolverservice} only the {\tt solve} method 
has been implemented to date.

If no options are specified, the default solver is used, depending on the problem characteristics 
(see Table~\ref{table:defaultsolvers} on p.\pageref{table:defaultsolvers}).\index{default solver}
%is to use {\tt Clp}\index{Clp@{\tt Clp}} for linear programs. 
%For continuous nonlinear models {\tt Ipopt}\index{Ipopt@{\tt Ipopt}} is used. 
%For mixed-integer linear models, {\tt Cbc}\index{Cbc@{\tt Cbc}} is used. 
%For mixed-integer nonlinear models  {\tt Bonmin}\index{Bonmin@{\tt Bonmin}} is used.  
If you wish to specify a specific solver, use the {\tt solver} option.   For example,  
since the test problem {\tt eastborne.mod} is a linear integer program, {\tt Cbc} is used by default. 
If instead you want to  use {\tt SYMPHONY}\index{COIN-OR projects!SYMPHONY@{\tt SYMPHONY}|(},
then you would pass a {\tt solver} option to the {\tt OSAmplclient.exe} solver as follows.%
\index{eastborne.mod@{\tt eastborne.mod}|)}

%\begin{verbatim}
%# tell OSAmplClient to use SYMPHONY instead of Cbc
\begin{verbatim}
option OSAmplClient_options "solver symphony";
\end{verbatim}
\index{COIN-OR projects!SYMPHONY@{\tt SYMPHONY}|)}

Valid values for the {\tt solver} option are installation-dependent.
%{\tt bonmin}, {\tt cbc}, {\tt clp}, {\tt couenne}, {\tt dylp}, {\tt symphony}, and {\tt vol}.   
The solver name in the {\tt solver} option is case insensitive.  

\ifbible
\medskip
It is possible to run {\tt OSAmplClient} in stand-alone mode directly from the command line. In this case, the first command line argument should be the name of a file in .nl format, e.g.,

\begin{verbatim}
OSAmplClient parincQuadratic
\end{verbatim}

\noindent (note that the file extension ({\tt .nl}) is omitted; this information is added by the ASL interface.)
\fi


\subsubdivision{Using OSAmplClient to Invoke an OS Solver Server}\label{section:remoteampl}

Next, assume that you have a large problem you want to solve on a remote solver. It is necessary 
to specify the location of the server solver as an option to OSAmplClient. 
The {\tt serviceLocation} option is used to specify the location of a solver server. 
In this case, the string of options for {\tt OSAmplClient\_options} is:

\begin{verbatim}
serviceLocation  http://xxx/OSServer/services/OSSolverService
\end{verbatim}
where {\tt xxx} is the URL for the server.  This string is used to replace the string `{\tt solver symphony}' in the previous example. 
The {\tt serviceLocation} option will send the problem to the %solver server at 
location {\tt http://xxx} and, assuming the remote executable is indeed found 
in the indicated folder, will start the executable.  


\medskip


However, each call 
\begin{verbatim}
option OSAmplClient_options
\end{verbatim}
is memoryless. That is, the options set in the last call will overwrite any options set in previous calls
and cause them to be discarded.  For instance, the sequence of option calls
\begin{verbatim}
option OSAmplClient_options "solver symphony";
option OSAmplClient_options "serviceLocation  
    http://xxx/OSServer/services/OSSolverService";
solve;
\end{verbatim}
will result in the default solver being called. 

If the intent is to use the SYMPHONY solver at the remote location, the option must be declared
as follows:

\begin{verbatim}
option OSAmplClient_options "solver symphony 
    serviceLocation http://xxx/OSServer/services/OSSolverService";
solve;
\end{verbatim}


For brevity we will omit the AMPL instruction
\begin{verbatim}
option OSAmplClient_options
\end{verbatim}
the double quotes and the trailing semicolon in the remaining examples.  

\medskip

Finally, the user may wish to pass options to the individual solver. This is done by specifying an options file.
(A sample options file, {\tt solveroptions.osol}\index{solveroptions.osol@{\tt solveroptions.osol}} is 
provided with this distribution).  The name of the options file is the value of the {\tt osol} option.
The string of options to {\tt OSAmplClient\_options} is now
\begin{verbatim}
serviceLocation http://xxx/OSServer/services/OSSolverService
osol solveroptions.osol
\end{verbatim}
This   {\tt solveroptions.osol}  file contains four solver options; two for {\tt Cbc}, one for {\tt Ipopt}, 
and one for {\tt SYMPHONY}\index{COIN-OR projects!SYMPHONY@{\tt SYMPHONY}}.
You can have any number of options. Note the format for specifying an option:
\begin{verbatim}
    <solverOption name="maxN" solver="cbc" value="5" />
\end{verbatim}
The attribute {\tt name} specifies that the option name is {\tt maxN} which is the maximum number of nodes 
allowed in the branch-and-bound tree, the {\tt solver} attribute specifies the name of the solver that the 
option should be applied to, and the {\tt value} attribute specifies the value of the option. 
As a second example, consider the specification
\begin{verbatim}
    <solverOption name="max_iter" solver="ipopt" type="integer" value="2000"/> 
\end{verbatim}
In this example we are specifying an iteration limit for {\tt Ipopt}.  Note the additional attribute 
{\tt type} that has value  {\tt integer}. The Ipopt solver requires specifying the data type 
(string, integer, or numeric) for its options.   Different solvers have different options, 
and we recommend that the user look at the documentation for the solver of interest in order to see 
which options are available.  
A good summary of options for COIN-OR solvers is 
%\url{http://www.coin-or.org/GAMSlinks/gamscoin.pdf}.
\url{http://www.gams.com/dd/docs/solvers/coin.pdf}.

If you examine the file {\tt solveroptions.osol} you will see that there is an XML tag  with the name
{\tt <solverToInvoke>} and that the solver given is {\tt symphony}.   
{\bf This has no effect on a local solve!} However, if this option file is paired with 

\begin{verbatim}
serviceLocation http://xxx/OSServer/services/OSSolverService
osol solveroptions.osol
\end{verbatim}
then in our reference implementation the remote solver service will parse the file {\tt solveroptions.osol}, find the {\tt <solverToInvoke>} tag and then pass the {\tt symphony} solver option to the {\tt OSSolverService} on the remote server.

\subsubdivision{AMPL Summary}

\begin{enumerate}
\item Tell  AMPL to use the OSAmplClient as the solver:

\begin{verbatim}
option solver OSAmplClient;
\end{verbatim}

\item Specify options to the OSAmplClient solver by using the AMPL command 

\begin{verbatim}
option OSAmplClient_options "(option string)";
\end{verbatim}

\item There are three possible options to specify:

\begin{itemize}
\item the location of the options file using  the {\tt osol} option;

\item the location of the remote server using   the {\tt serviceLocation} option;

\item the name of the solver using the  {\tt solver} option; valid values for this option  are 
%{\tt clp}, {\tt cbc},  {\tt dylp},  {\tt ipopt}, {\tt bonmin},   {\tt couenne},  and  {\tt symphony}
installation-dependent. 
For details, see Table~\ref{table:configurations} on page~\pageref{table:configurations} 
and the discussion in Section~\ref{section:OSSolverServiceInputParameters}. 

\end{itemize}

These three options behave {\it exactly like} the {\tt solver}, {\tt serviceLocation}, and {\tt osol} options used by the {\tt OSSolverService} described in  Section \ref{section:commandlineparser}.
Note that the {\tt solver} option only has an effect with a local solve; 
if the user wants to invoke a specific solver with a remote solve, then this must be done in the OSoL file using the {\tt <solverToInvoke>} element.

\item  The options given to {\tt OSAmplClient\_options}  can be given in any order.

\item If no solver is specified using {\tt OSAmplClient\_options},  the default solver is used.
(For details see Table~\ref{table:defaultsolvers}).\index{default solver}

\item A remote solver is called if and only if the {\tt serviceLocation} option is specified.

\end{enumerate}

\index{OSAmplClient@{\tt OSAmplClient}|)}
\index{AMPL|)}



\subdivision{GAMS and Optimization Services}\label{section:gamslinks}

\index{GAMS|(}

This section pertains to GAMS version 23.8 (and above) that now includes support for OS.  
Here we describe the GAMS  implementation of Optimization Services.  We assume that the user has installed GAMS.

In GAMS, OS is implemented through the {\tt CoinOS} solver that is packaged with GAMS.      
The GAMS {\tt CoinOS} solver is really a {\it solver interface} that links to the OS library.
At present the GAMS  {\tt CoinOS} solver does not support local calls, but it can be used to make
remote calls to an {\tt OSSolverService} executable on a remote server. How this is done is the topic of the next section.



\subsubdivision{Using GAMS  to Invoke a Remote OS Solver Service}\label{section:gamsremote}

We now describe how to call  a remote OS   solver service using the GAMS {\tt CoinOS}.  Before proceeding, 
it is important to emphasize that when calling a remote OS solver service, different sets of solvers may be supported, even for the same version of the OS solver service. 
For example, the remote 
implementation may provide access to solvers such as {\tt SYMPHONY}, {\tt Couenne}, {\tt Glpk} and {\tt DyLP}.  
There are several reason why you might wish to use a remote OS solver service. 

\begin{itemize}
\item Have access to a faster machine.

\item  Be able to  submit jobs to run in asynchronous mode -- submit your job,  turn off your laptop,  
and check later to see if the job ran.

\item Call several additional solvers (e.g., {\tt SYMPHONY}, {\tt Couenne}, {\tt Glpk} and {\tt DyLP}).
Note, however, that not all solvers may be available available locally (especially Glpk) may not be available for a remote call.

\end{itemize}

We will illustrate several possible calls with the sample GAMS file {\tt eastborne.gms} which found in the
{\tt  data/gamsFiles} directory. We assume that this file exists in the current directory and that the GAMS executable is found in the search path. The command to execute at the command line would then be

\begin{verbatim} 
gams eastborne.gms MIP=CoinOS optfile=1
\end{verbatim}

The server name ({\tt CoinOS}) is case-insensitive and could equally well have been written as 
``{\tt MIP=coinos}'' or ``{\tt MIP=COINOS}''. Moreover, the file {\tt eastborne.gms} contains the directive

\begin{verbatim}
Option MIP = CoinOS;
\end{verbatim}

\noindent and hence the option {\tt MIP=CoinOS} could have been omitted from the command line.

Since the solver is named {\tt CoinOS}, the options file pointed to by the last part of the command
({\tt optfile=1}) should be named {\tt CoinOS.opt}. In general multiple option files are possible, and the GAMS convention is as follows:

{\tt optfile=1} corresponds to {\tt CoinOS.opt}

{\tt optfile=2} corresponds to {\tt CoinOS.op2}

{\ldots}

{\tt optfile=99} corresponds to {\tt CoinOS.o99}
\medskip
It is important to distinguish between the option files for GAMS just mentioned and the  option file (in OSoL format) passed to the OS solver server (see below).
We now explain the valid options that can go into a GAMS option file when using the CoinOS solver. 
The options are

\vskip 8pt
\noindent{\tt service (string)}: Specifies the URL of  the COIN-OR solver service. 
This option is required in order to direct the remote call appropriately.
\vskip 8pt
Use the following value for this option.
\begin{verbatim}
service http://74.94.100.129:8080/OSServer/services/OSSolverService
\end{verbatim}


\iffalse
\vskip 8pt
\noindent {\tt solver  (string)}:   Specifies the solver that is used to solve an instance. 
Valid values are {\tt clp},  {\tt cbc}, {\tt glpk}, {\tt ipopt},  and {\tt bonmin}.  
If a solver name is specified that is not recognized, the default solver for the problem type is used.  
The value for the solver option is case insensitive. 
For example, if the file {\tt CoinOS.opt} contains the two lines
\begin{verbatim}
service http://74.94.100.129:8080/OSServer/services/OSSolverService
solver glpk
\end{verbatim}
then executing
\begin{verbatim}
gams.exe eastborne.gms optfile 1
\end{verbatim}
will result in  using {\tt Glpk}  to solve the problem.   
\fi

\vskip 8pt
\noindent {\tt writeosil  (string)}:  If this option is used, GAMS will write the optimization instance 
to file {\tt (string)} in    OSiL   format.
\vskip 8pt

\vskip 8pt
\noindent {\tt writeosrl  (string)}:  If this option is used, GAMS will write the result of the optimization 
to file {\tt (string)} in OSrL  format.
\vskip 8pt

The options just described are options for the GAMS modeling language.  
It is also possible to pass options directly to the COIN-OR solvers by using the {\tt OS} interface.
This is done by passing the name of an options file that conforms to the  OSoL  standard.  
%See \url{http://projects.coin-or.org/OS}  for information on Optimization Services.  
The option

\vskip 8pt
\noindent {\tt readosol  (string)}  specifies the name of an OS option  file in OSoL format that is 
given to the solver.  {\bf Note well:} The file  {\tt CoinOS.opt} is an option  file for GAMS but the GAMS option 
{\tt readosol} in the GAMS options file  is specifying the name of an OS options file. 
\vskip 8pt
The file {\tt solveroptions.osol} is contained in the OS distribution in the {\tt osolFiles} directory   
in the {\tt data} directory. This file contains four solver options; two for {\tt Cbc}, one for {\tt Ipopt},
and one for {\tt SYMPHONY} (which is available for remote server calls, but not locally).  
You can have any number of options. Note the format for specifying an option:
\begin{verbatim}
    <solverOption name="maxN" solver="cbc" value="5" />
\end{verbatim}
The attribute {\tt name} specifies that the option name is {\tt maxN} which is the maximum number of nodes 
allowed in the branch-and-bound tree, the {\tt solver} attribute specifies the name of the solver to which
the option should be applied, and the {\tt value} attribute specifies the value of the option. 

Default solver values are present, depending on the problem characteristics. For more details, consult 
Table~\ref{table:defaultsolvers} (p.\pageref{table:defaultsolvers}).
In order to control the solver used, it is necessary to specify the name of the solver
inside the XML tag {\tt <solverToInvoke>}. The example  {\tt solveroptions.osol} file contains the XML tag
\begin{verbatim}
    <solverToInvoke>symphony</solverToInvoke>
\end{verbatim}

\iffalse
If, for example,  the {\tt CoinOS.opt} file is
\begin{verbatim}
solver ipopt
service http://74.94.100.129:8080/OSServer/services/OSSolverService
readosol  solveroptions.osol
writeosrl temp.osrl
\end{verbatim}
then {\tt Ipopt} is ignored as a solver option and the remote server uses the {\tt  SYMPHONY} solver.
\fi  
Valid values for the remote solver service specified in the {\tt <solverToInvoke>} tag are 
installation dependent; the solver service at 
{\tt http://74.94.100.129:8080/OSServer/services/OSSolverService} accepts
{\tt clp},  
{\tt cbc},  {\tt dylp}, {\tt glpk}, {\tt ipopt}, {\tt bonmin},   {\tt couenne},  {\tt symphony}, and 
{\tt vol}.  

%If the problem is solved using a remote solver service the value specified by the 
%GAMS {\tt solver} option is irrelevant and ignored. 

\medskip



By default, the call to the server is a {\it synchronous} call. The GAMS process will wait for the result 
and then display the result. This may not be desirable when solving large optimization models.  
The user may wish to submit a job, turn off his or her computer,  and then check at a later date to see 
if the job is finished.  In order to use the remote solver service in this fashion, i.e., 
{\it asynchronously}, it  is necessary to use the  {\tt service\_method} option.

\vskip 8pt
\noindent {\tt service\_method (string)} specifies the method to execute on a server.  
Valid values for this option are {\tt solve}, {\tt getJobID}, {\tt send}, {\tt knock}, {\tt retrieve}, 
and {\tt kill}. We explain how to use each of these.
\vskip 8pt
The default value of {\tt service\_method} is {\tt solve.} A {\tt solve} invokes the remote service 
in synchronous mode. When using the {\tt solve} method you can optionally specify a set of solver options 
in an OSoL file  by using the {\tt readosol} option. The  remaining values for the {\tt service\_method} 
option are used for an asynchronous call.  We illustrate them in the order in which they would most 
logically be executed. 

\vskip 8pt
\noindent {\tt service\_method getJobID}: When working in asynchronous mode, the server needs to 
uniquely identify each job. The {\tt getJobID} service method will result in the server returning 
a unique job ID. For example if the following {\tt CoinOS.opt} file is used
\vskip 8pt
\begin{verbatim}
service http://74.94.100.129:8080/OSServer/services/OSSolverService
service_method getJobID
\end{verbatim}
with the command
\begin{verbatim}
gams.exe eastborne.gms optfile=1
\end{verbatim}
the user will see a rather long job ID returned to the screen as output. Assume that the job id returned 
is {\tt coinor12345xyz}. This job ID is used to submit a job to the server with the {\tt send} method.
Any job ID can be sent to the server as long as it has not been used before.  

\vskip 8pt
\noindent {\tt service\_method send}: When working in asynchronous mode, use the {\tt send} service method 
to submit a job. When using  the {\tt send} service method a job ID is required. An options file
must be present and must specify a  job ID that has not been used before.  Assume that in the file {\tt CoinOS.opt}  we specify 
the options:
\vskip 8pt
\begin{verbatim}
service http://74.94.100.129:8080/OSServer/services/OSSolverService
service_method send
readosol sendWithJobID.osol
\end{verbatim}
The {\tt sendWithJobID.osol} options file is identical to the {\tt solveroptions.osol} options file except 
that it has an additional XML tag:
\begin{verbatim}
    <jobID>coinor12345xyz</jobID> 
\end{verbatim}
We then execute
\vskip 8pt
\begin{verbatim}
gams.exe eastborne.gms optfile=1
\end{verbatim}
If all goes well, the response to the above command should  be: 
``Problem instance successfully sent to OS service''. 
At this point the server will schedule the job and work on it. It is possible to turn off 
the user computer at this point. At some point the user will want to know if the job is finished. 
This is accomplished using the {\tt knock} service method.
\vskip 8pt
\noindent {\tt service\_method knock}: When working in asynchronous mode, this is used to check the status 
of a job.  Consider the following {\tt CoinOS.opt} file:
\vskip 8pt
\begin{verbatim}
service http://74.94.100.129:8080/OSServer/services/OSSolverService
service_method knock
readosol sendWithJobID.osol 
readospl knock.ospl
writeospl knockResult.ospl
\end{verbatim}
The {\tt knock} service method requires two  inputs. The first input is the name of an options file, 
in this case {\tt sendWithJobID.osol}, specified through the {\tt readosol} option. In addition, a file 
in OSpL format is required. You can use the {\tt knock.opsl} file provided in the binary distribution. 
This file name is specified using the {\tt readospl} option. If no job ID is specified in the OSoL file 
then the status of all jobs on the server will be returned in the file specified by the {\tt writeospl} 
option. If a job ID is specified in the OSoL file, then only information on the specified job ID is 
returned in the file specified by the {\tt writeospl} option.  In this case the file name is 
{\tt knockResult.ospl}. We then execute
\vskip 8pt
\begin{verbatim}
gams.exe eastborne.gms optfile=1
\end{verbatim}
The file {\tt knockResult.ospl} will contain information similar to the following:
\begin{verbatim}
    <job jobID="coinor12345xyz">
        <state>finished</state>
        <serviceURI>http://192.168.0.219:8443/os/OSSolverService.jws</serviceURI>
        <submitTime>2009-11-10T02:13:11.245-06:00</submitTime>
        <startTime>2009-11-10T02:13:11.245-06:00</startTime>
        <endTime>2009-11-10T02:13:12.605-06:00</endTime>
        <duration>1.36</duration>
    </job>
\end{verbatim}
Note that the job is complete as indicated in the {\tt <state>} tag. It is now time to actually retrieve 
the job solution.  This is done with the {\tt retrieve} method.
\vskip 8pt
\noindent {\tt service\_method retrieve}: When working in asynchronous mode, this method is used 
to retrieve the job solution. It is necessary when using {\tt retrieve} %{\tt knock} ???
to specify an options file and in that options file specify a job ID.   
Consider the following {\tt CoinOS.opt} file:
\vskip 8pt
\begin{verbatim}
service http://74.94.100.129:8080/OSServer/services/OSSolverService
service_method retrieve
readosol sendWithJobID.osol
writeosrl answer.osrl
\end{verbatim}
When we then execute
\vskip 8pt
\begin{verbatim}
gams.exe eastborne.gms optfile=1
\end{verbatim}
the result is written to the file {\tt answer.osrl}. 

Finally there is a {\tt kill} service method which is used to kill a job that was submitted by mistake 
or is running too long on the server. 
\vskip 8pt
\noindent {\tt service\_method kill:} When working in asynchronous mode, this method is used to terminate 
a job. You should specify an OSoL  file containing the job ID by using the {\tt readosol} option.
\vskip 8pt

\iffalse
\subsubdivision{Using GAMS to Invoke the Local OS Solver Service \tt CoinOS}\label{section:gamslocal}

   
The GAMS  {\tt CoinOS} solver is really a {\it solver interface} and is linked through the OS library to the 
following COIN-OR solvers: {\tt Bonmin}, {\tt Cbc}, {\tt Clp},  {\tt Glpk}, and {\tt Ipopt}. 
Think of {\tt CoinOS} as a {\it metasolver}.    As an example (we assume a Windows operating system 
and use the .exe extension), consider:

\begin{verbatim}
gams.exe eastborne.gms MIP=CoinOS
\end{verbatim}
The solver name {\tt CoinOS} is not case sensitive and 
\begin{verbatim}
gams.exe eastborne.gms MIP=coinos
\end{verbatim}
will also work.  In addition, if
\begin{verbatim}
Option MIP = CoinOS ;
\end{verbatim}
is present in the GAMS file, then writing {\tt MIP=CoinOS} on the command line is unnecessary.
Since {\tt Option MIP = CoinOS;} is present in the GAMS model file {\tt eastborne.gms}, 
we will not specify it explicitly on the command line in the ensuing discussion. To summarize,
\begin{verbatim}
gams.exe eastborne.gms 
\end{verbatim}
is equivalent to the two versions of the command given previously.  Executing any of the commands will 
result in the model being solved on the local machine using the COIN-OR solver {\tt Cbc}, the default solver 
for 
%continuous linear models (LP and RMIP), {\tt CoinOS} chooses {\tt Clp}. For continuous nonlinear 
%models (NLP, DNLP, RMINLP, QCP, RMIQCP), {\tt Ipopt} is the default solver. For 
mixed-integer linear models (MIP).
%,  {\tt Cbc} is the default solver. For mixed-integer nonlinear models (MIQCP, MINLP), 
%{\tt Bonmin} is the default solver.

It is possible to control which solver is selected by {\tt CoinOS}.    This is done by providing an {\it options file}  to  GAMS.   
Since the solver is named {\tt  CoinOS}, the options file should  be named {\tt CoinOS.opt}  (the file name is not case sensitive)
and the command line call is 
\begin{verbatim}
gams.exe eastborne.gms optfile 1
\end{verbatim}
Calling multiple GAMS options files uses the convention
\begin{verbatim}
optfile=1 corresponds to CoinOS.opt
optfile=2 corresponds to CoinOS.op2
...
optfile=99 corresponds to CoinOS.o99
\end{verbatim}

We now explain the valid options that can go into a GAMS option file when using the {\tt CoinOS} solver.  They are:

\vskip 8pt
\noindent {\tt solver  (string)}:   Specifies the solver that is used to solve an instance. 
Valid values are {\tt clp},  {\tt cbc}, {\tt glpk}, {\tt ipopt},  and {\tt bonmin}.  
If a solver name is specified that is not recognized, the default solver for the problem type is used.  
The value for the solver option is case insensitive. 
For example, if the file {\tt CoinOS.opt} contains a single line
\begin{verbatim}
solver glpk
\end{verbatim}
then executing
\begin{verbatim}
gams.exe eastborne.gms optfile 1
\end{verbatim}
will result in  using {\tt Glpk}  to solve the problem.   


\vskip 8pt
\noindent {\tt writeosil  (string)}:  If this option is used, GAMS will write the optimization instance 
to file {\tt (string)} in    OSiL   format.
\vskip 8pt

\vskip 8pt
\noindent {\tt writeosrl  (string)}:  If this option is used, GAMS will write the result of the optimization 
to file {\tt (string)} in OSrL  format.
\vskip 8pt

The options just described are options for the GAMS modeling language.  
It is also possible to pass options directly to the COIN-OR solvers by using the {\tt OS} interface.
This is done by passing the name of an options file that conforms to the  OSoL  standard.  
%See \url{http://projects.coin-or.org/OS}  for information on Optimization Services.  
The option

\vskip 8pt
\noindent {\tt readosol  (string)}  specifies the name of an OS option  file in OSoL format that is 
given to the solver.  Note: The file  {\tt CoinOS.opt} is an option  file for GAMS but the GAMS option 
{\tt readosol} in the GAMS options file  is specifying the name of an OS options file. 
\vskip 8pt
The file {\tt solveroptions.osol} is contained in the OS distribution in the {\tt osolFiles} directory   
in the {\tt data} directory. This file contains four solver options; two for {\tt Cbc}, one for {\tt Ipopt},
and one for {\tt SYMPHONY} (which is available for remote server calls, but not locally).  
You can have any number of options. Note the format for specifying an option:
\begin{verbatim}
    <solverOption name="maxN" solver="cbc" value="5" />
\end{verbatim}
The attribute {\tt name} specifies that the option name is {\tt maxN} which is the maximum number of nodes 
allowed in the branch-and-bound tree, the {\tt solver} attribute specifies the name of the solver to which
the option should be applied, and the {\tt value} attribute specifies the value of the option. 

As a second example, consider the specification
\begin{verbatim}
    <solverOption name="max_iter" solver="ipopt" type="integer" value="2000"/> 
\end{verbatim}
In this example we are specifying an iteration limit for {\tt Ipopt}.  Note the additional attribute 
{\tt type} that has value  {\tt integer}. The Ipopt solver requires specifying the data type 
(string, integer, or numeric) for its options.   For a list of options that solvers take, 
see the file
\begin{verbatim}
docs/solvers/coin.pdf
\end{verbatim}
inside the GAMS directory. 
An up-to-date online version of this list is available at \url{http://www.coin-or.org/GAMSlinks/gamscoin.pdf}.

\fi

\subsubdivision{GAMS Summary:}\label{section:gamssummary}


\begin{enumerate}

\item[1.]   In order to use OS with GAMS you can either specify {\tt CoinOS} as an option to GAMS 
at the command line,
\begin{verbatim}
gams eastborne.gms MIP=CoinOS
\end{verbatim}
or you can  place the statement {\tt Option ProblemType = CoinOS;} somewhere in the model {\it before} 
the {\tt Solve} statement in the GAMS file.


\item[2.]   If no options are given, then the model will be solved locally using the default solver 
(see Table~\ref{table:defaultsolvers} on p.\pageref{table:defaultsolvers}).
%and {\tt Clp} will be used for 
%linear programs, {\tt Cbc} for integer linear programs, {\tt Ipopt} for continuous nonlinear programs, 
%and {\tt Bonmin} for nonlinear integer programs.

\item[3.] In order to control behavior (for example, whether a local or remote solver is used)  an options
 file,  {\tt CoinOS.opt}, must be used as follows

\begin{verbatim}
gams.exe  eastborne.gms optfile=1
\end{verbatim}

\item[4.]  The  {\tt CoinOS.opt} file is used to specify {\it eight potential options}:


\begin{itemize}
\item {\tt service (string)}: using the COIN-OR solver server; this is done by giving the option

\begin{verbatim}
service  http://74.94.100.129:8080/OSServer/services/OSSolverService
\end{verbatim}


\item  {\tt readosol (string)}: whether or not to send the solver an options file; this is done by 
giving the option
\begin{verbatim}
readosol  solveroptions.osol
\end{verbatim}


\item   {\tt solver (string)}: if a local solve is being done,  a specific solver is specified by 
the option
\begin{verbatim}
solver solver_name
\end{verbatim}

Valid values are {\tt clp},  {\tt cbc}, {\tt glpk}, {\tt ipopt} and {\tt bonmin}. %  and {\tt couenne}.  
When the COIN-OR solver service is being used, the only way to specify the solver to use is through 
the {\tt <solverToInvoke>} tag in an OSoL file. In this case the valid values for the solver are  
{\tt clp}, {\tt cbc}, {\tt dylp}, {\tt glpk}, {\tt ipopt}, {\tt bonmin}, {\tt couenne}, {\tt symphony}
and {\tt vol}.



\item  {\tt writeosrl (string)}:  the solution result can be put into an OSrL file by specifying the option

\begin{verbatim}
writeosrl  osrl_file_name
\end{verbatim}



\item    {\tt writeosil (string)}:   the optimization instance  can be put into an OSiL file by specifying 
the option



\begin{verbatim}
writeosil  osil_file_name
\end{verbatim}


\item {\tt writeospl (string):} Specifies the name of an OSpL  file in which the answer from the 
{\tt knock} or {\tt kill} method is written, e.g.,

\begin{verbatim}
writeospl  write_ospl_file_name
\end{verbatim}


\item {\tt readospl (string):} Specifies the name of an OSpL  file that the {\tt knock} method 
sends to  the server

\begin{verbatim}
readospl  read_ospl_file_name
\end{verbatim}

\item {\tt service\_method (string)}: Specifies the method to execute on a server.  Valid values 
for this option are {\tt solve}, {\tt getJobID}, {\tt send}, {\tt knock}, {\tt retrieve}, and {\tt kill}.

\end{itemize}

\item[5.]  If an OS options file is passed to the GAMS {\tt CoinOS} solver using the GAMS  {\tt CoinOS} option      {\tt readosol}, then GAMS does not interpret  or act on any options in this file. The options in the OS options file are passed directly to either: i) the default local solver, ii) the local solver specified by the  GAMS {\tt CoinOS}  option {\tt solver}, or iii)  to the remote OS solver service if one is specified by the GAMS  {\tt CoinOS} option {\tt service.}

\end{enumerate}

\index{GAMS|)}

\ifruncode\else    % the matlab interface requires the user to compile stuff

\subdivision{MATLAB:  Using MATLAB to Build and Run OSiL Model Instances}\label{section:usingmatlab}

\index{MATLAB|(}
MATLAB has powerful matrix generation and manipulation routines. This section is for users who wish to use MATLAB to generate the matrix coefficients for linear or quadratic programs and use the OS library to call a solver and get the result back. Using MATLAB with OS requires the user to compile a file {\tt OSMatlabSolverMex.cpp} into a MATLAB executable file (these files will have a {\tt .mex} extension) after compilation. This executable file is linked to the OS library and works through the MATLAB API to communicate with the OS library. 



The OS MATLAB application differs from the other applications in the {\tt OS/applications} folder in that makefiles are not used.  The file 
\begin{verbatim}
OS/applications/matlab/OSMatlabSolverMex.cpp
\end{verbatim}
must be compiled inside the MATLAB command window.  Building the OS MATLAB application requires the following steps. 


\begin{enumerate}[{\bf Step 1:}]



\item{}   The MATLAB installation contains a file {\tt mexopts.sh} (UNIX) or {\tt mexopts.bat}  (Windows) that must be edited.   This file typically resides  in the {\tt bin} directory of the MATLAB application.    This file  contains compile and link options that must be properly set.   Appropriate paths to header files and libraries must be set.  This discussion is based on the assumption that the user has either done a  {\tt make install} for the OS project or has downloaded a binary archive of the OS project. In either case there will be an {\tt include} directory with the necessary header files and a {\tt lib} directory with the necessary libraries for linking. 

First edit   the {\tt CXXFLAGS} option  to point to  the header files in the {\tt cppad} directory and the {\tt include} directory in the project root. For example, it  should look like:
\begin{verbatim}
CXXFLAGS='-fno-common -no-cpp-precomp -fexceptions
    -I/Users/kmartin/Documents/files/code/cpp/OScpp/COIN-OS/
    -I/Users/kmartin/Documents/files/code/cpp/OScpp/COIN-OS/include'
\end{verbatim}

Next edit the {\tt CXXLIBS} flag so that the OS and supporting libraries are included. For example, it should look like the following\footnote{The libraries to include in CXXLIBS depends upon which projects were compiled with OS.} on a MacIntosh:

\begin{verbatim}
CXXLIBS="$MLIBS -lstdc++ -L/Users/kmartin/coin/os-trunk/vpath/lib 
-lOS -lbonmin -lIpopt -lOsiCbc -lOsiClp -lOsiSym -lOsiVol
-lOsiDylp -lCbc -lCgl -lOsi -lClp  -lSym -lVol -lDylp 
-lCoinUtils -lCbcSolver  -lcoinmumps -ldl -lpthread 
/usr/local/lib/libgfortran.dylib -lgcc_s.10.5 -lgcc_ext.10.5 -lSystem -lm 
\end{verbatim}

{\bf Important:} It has been the authors' experience that setting the necessary MATLAB compiler and linker options to build the {\tt mex} can be tricky.  We include in
\begin{verbatim}
OS/applications/matlab/macOSXscript.txt
\end{verbatim}
the exact options that work on a 64 bit Mac with MATLAB release R2009b.

\item{}  Build the MATLAB executable file. Start MATLAB and in the MATLAB command window connect to the directory {\tt OS/examples/matlab} which  contains the file 

\begin{verbatim}
OSMatlabSolverMex.cpp
\end{verbatim}

\item{} Execute the command:

\begin{verbatim}
mex -v OSMatlabSolverMex.cpp
\end{verbatim}

On a 64 bit machine the command should be

\begin{verbatim}
mex -v -largeArrayDims OSMatlabSolverMex.cpp
\end{verbatim}

The name of the resulting executable is system dependent. 
On an Intel MAC OS X 64 bit chip the name will be  {\tt OSMatlabSolver.mexmaci64}, 
on a Windows system it is {\tt OSMatlabSolver.mexw32}.  



\item{}  Set the MATLAB path to include the directory {\tt  OS/applications/matlab}  (or more generally, the directory with the {\tt mex} executable).


\item{}   In the MATLAB command window, connect to the directory {\tt OS/data/matlabFiles}. Run either of the MATLAB
files {\tt markowitz.m} or {\tt parincLinear.m}.  The result should be displayed in the MATLAB browser window.

\end{enumerate}


To use the {\tt OSMatlabSolver} it is necessary to put the coefficients  from a linear, integer, or quadratic problem into MATLAB arrays.   We illustrate for the linear program:

\begin{alignat}{2}
& \mbox{Minimize} & \quad
10 x_{1} + 9 x_{2}\label{eq:parinobj}\\
& \mbox{Subject to} & \quad .7x_{1} + x_{2}  &\le 630  \label{eq:parinccon1}\\
& & .5x_{1} + (5/6) x_{2} &\le 600 \label{eq:parinccon2}\\
& &  x_{1} + (2/3) x_{2} &\le 708 \label{eq:parinccon3}\\
& & .1x_{1} + .25 x_{2} &\le 135 \label{eq:parinccon4}\\
& & x_{1}, x_{2} &\ge 0 \label{eq:parincnonneg}
\end{alignat}

The MATLAB representation of this problem in MATLAB arrays is
\begin{verbatim}
% the number of constraints
numCon = 4;
% the number of variables
numVar = 2;
% variable types
VarType='CC';
% constraint types
A = [.7  1; .5  5/6; 1   2/3  ; .1   .25];
BU = [630 600  708  135];
BL = [];
OBJ = [10  9];
VL = [-inf -inf];
VU = [];
ObjType = 1;
% leave Q empty if there are no quadratic terms
Q = [];
prob_name = 'ParInc Example'
password = '';
%
%
%the solver
solverName = 'ipopt';
%the remote service address
%if left empty we solve locally -- must solve locally for now
serviceLocation='';
% now solve
callMatlabSolver( numVar, numCon, A, BL, BU, OBJ, VL, VU, ObjType, ...
    VarType, Q, prob_name, password, solverName, serviceLocation)
\end{verbatim}
This example m-file is in the {\tt data} directory and is file {\tt parincLinear.m}. Note that in addition to the problem formulation
we can specify which solver to use through the {\tt solverName} variable.  If solution with a remote solver is desired
this can be specified with the {\tt serviceLocation} variable.  If the {\tt serviceLocation} is left empty, i.e.,
\begin{verbatim}
serviceLocation='';
\end{verbatim}
then a local solver is used. In this case  it is crucial that the appropriate solver is linked in with the {\tt matlabSolver}
executable using the {\tt CXXLIBS} option.


The data directory  also contains the m-file  {\tt template.m} which contains extensive comments about how to formulate
the problems in MATLAB.   The user can edit {\tt template.m} as necessary and create a new instance.




 A second example which is a quadratic problem is given in Section~\ref{section:usingmatlab}.
The appropriate MATLAB m-file is {\tt markowitz.m} in the {data/matlabFiles} directory.
The problem consists in investing  in a number of stocks. The expected returns and risks
(covariances) of the stocks are known. Assume that the decision variables $x_i$
represent the fraction of wealth invested in stock~$i$ and that no stock can have
more than 75\% of the total wealth. The problem then is to minimize the total risk
subject to a budget constraint and a lower bound on the expected portfolio return.

Assume that there are three stocks (variables) and two constraints (not counting the upper limit  %investment
of .75 on the investment variables).


\begin{verbatim}
% the number of constraints
numCon = 2;
% the number of variables
numVar = 3;
\end{verbatim}



All the variables are continuous:


\begin{verbatim}
VarType='CCC';
\end{verbatim}


Next define the constraint upper and lower bounds. There are two constraints, an equality  constraint (an $=$) and a lower bound on portfolio return of .15 (a $\ge$). These two constraints are expressed as



\begin{verbatim}
BL = [1   .15];
BU = [1  inf];
\end{verbatim}



The variables are nonnegative and have upper limits of .75 (no stock can comprise more than 75\% of the portfolio).  This is written as




\begin{verbatim}
VL = [];
VU = [.75 .75 .75];
\end{verbatim}



There are no nonzero linear coefficients in the objective function, but the objective function vector must always be defined and the number of components of this vector is the number of variables.



\begin{verbatim}
OBJ = [0 0 0 ]
\end{verbatim}


 Now the linear constraints.   In the model the two linear constraints are
 \begin{eqnarray*}
 x_{1} + x_{2} + x_{3} &=& 1 \\
 0.3221 x_{1} +   0.0963x_{2} +    0.1187x_{3}  &\ge& .15
 \end{eqnarray*}



 These are expressed as



 \begin{verbatim}
 A = [ 1 1 1  ;
  0.3221   0.0963   0.1187 ];
 \end{verbatim}


Now for the quadratic terms. The only quadratic terms are in the objective function. The objective function is


\begin{eqnarray*}
\min  0.425349694 x_{1}^{2} +  0.445784443 x_{2}^{2} + 0.231430983 x_{3}^{2} + 2 \times 0.185218694 x_{1} x_{2} \\
+ 2 \times 0.139312545 x_{1} x_{3} + 2 \times 0.13881692 x_{2} x_{3}
\end{eqnarray*}


To represent quadratic terms MATLAB uses an array, here denoted $Q$, which has four rows, and a column for each quadratic term. 
In this example there are six quadratic terms. The first row of $Q$ is the row index where the terms appear. By convention, 
the objective function has index -1, and constraints are counted starting at 0.  The first row of $Q$ is


 \begin{verbatim}
 -1 -1 -1 -1 -1 -1
 \end{verbatim}

The second row of $Q$ is the index of the first variable in the quadratic term. We use zero based counting.  
Variable $x_{1}$ has index 0, variable  $x_{2}$ has index 1, and variable $x_{3}$ has index 2.  
Therefore, the second row of $Q$ is



\begin{verbatim}
0 1 2 0 0 1
\end{verbatim}



The third row of $Q$ is the index of the second variable in the quadratic term.   Therefore, the third row of $Q$ is



\begin{verbatim}
0 1 2 1 2 2
\end{verbatim}

Note that terms such as $x_1^2$ are treated as $x_1*x_1$ and that mixed terms such as $x_2x_3$ could be given in either order.

The last (fourth) row is the coefficient. Therefore, the fourth row reads





\begin{verbatim}
.425349654  .445784443  .231430983   .370437388  .27862509   .27763384
\end{verbatim}


The full array is



\begin{verbatim}
Q = [ -1 -1 -1 -1 -1 -1;
      0 1 2 0 0 1 ;
      0 1 2 1 2 2;
      .425349654  .445784443  .231430983   .370437388  .27862509   .27763384
    ];
\end{verbatim}


Finally, name the problem, specify the solver (in this case {\tt ipopt}), the service address (and password if required by the service), and call the solver.



\begin{verbatim}
% replace Template with the name of your  problem
prob_name = 'Markowitz Example from Anderson, Sweeney, Williams, and Martin';
password = '';
%
%the solver
solverName = 'ipopt';
%the remote service service address
%if left empty we solve locally -- must solve locally for now
serviceLocation='';
% now solve
OSCallMatlabSolver( numVar, numCon, A, BL, BU, OBJ, VL, VU, ObjType, VarType, ...
     Q, prob_name, password, solverName, serviceLocation)
\end{verbatim}
\index{MATLAB|)}

\fi

